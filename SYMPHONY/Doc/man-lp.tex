%===========================================================================%
%                                                                           %
% This file is part of the documentation for the SYMPHONY MILP Solver.      %
%                                                                           %
% SYMPHONY was jointly developed by Ted Ralphs (tkralphs@lehigh.edu) and    %
% Laci Ladanyi (ladanyi@us.ibm.com).                                        %
%                                                                           %
% (c) Copyright 2000-2006 Ted Ralphs. All Rights Reserved.                  %
%                                                                           %
% SYMPHONY is licensed under the Common Public License. Please see          %
% accompanying file for terms.                                              %
%                                                                           %
%===========================================================================%

%%%%%%%%%%%%%%%%%%%%%%%%%%%%%%%%%%%%%%%%%%%%%%%%%%%%%%%%%%%%%%%%%%%%%%%%%%%%%
\subsection{LP module callbacks}
\label{user-written-lp}

\subsubsection{Data Structures}
\label{data_structures}

We first describe a few structures that are used to pass
data into and out of the user functions of the LP module.

\begin{description}

\begin{htmlonly}
\paragraph{MIPdesc}
\end{htmlonly}
%\begin{latexonly}
\firstfuncdef{MIPdesc}
%\end{latexonly}
\label{MIPdesc}

One of the few internally defined data structures that the user has to deal
with frequently is the \texttt{MIPdesc} data structure, which holds the data
needed to describe a MILP. This data structure is used by SYMPHONY for two
purposes. First, it is used to store the description of a generic MILP that is
either read in from an MPS or AMPL file. More importantly, it is the data
structure the user must use to pass the subproblem descriptions back to
SYMPHONY at the time a new search tree node is created in the function
\hyperref{{\tt user\_create\_subproblem()}} {\ptt {user\_create\_subproblem()}
(see Section }{)}{user_create_subproblem}. The structure has 14 fields (listed
below) that must be filled out to describe a subproblem (some fields are
optional). \\
\\
A subproblem is a mixed-integer program defined by a matrix of constraints, an
objective function, bounds on both the right hand side values of the
constraints and the variables, an array indicating which variables are
integer, and (optionally) an array of column names that allows SYMPHONY to
report the solution back in terms of column names instead of user indices. If
the subproblem has \texttt{n} variables and \texttt{m} constraints, the
constraints are given by a constraint coefficient matrix of size \texttt{m}
$\times$ \texttt{n} (described in the next paragraph). The sense of each
constraint, the right hand side values and bounds on the right hand side
(called {\em range}) are vectors are of size \texttt{m}. The objective
function coefficients and the lower and upper bounds on the variables are
vectors of length \texttt{n}. The sense of each constraint can be either 'L'
($\leq$), 'E' ($=$), 'G' ($\geq$) or 'R' (ranged). For non-ranged rows the
range value is {\tt 0}, for a ranged row the range value must be non-negative
and the constraint means that the row activity level has to be between the
right hand side value and the right hand side increased by the range value. \\
\\
Since the coefficient matrix is very often sparse, only the nonzero entries
are stored. Each entry of the matrix has a column index, a row index and a
coefficient value associated with it. A matrix is specified in the form of the
three arrays \texttt{matval}, \texttt{matind}, and \texttt{matbeg}. The array
\texttt{matval} contains the values of the nonzero entries of the matrix in
\emph{column order}; that is, all the entries for the $0^{th}$ column come 
first, then the entries for the $1^{st}$ column, etc. The row index
corresponding to each entry of \texttt{matval} is listed in \texttt{matind}
(both of them are of length \texttt{nz}, the number of nonzero entries in the
matrix). Finally,
\texttt{matbeg} contains the starting positions of each of the columns in 
\texttt{matval} and \texttt{matind}. Thus, \texttt{matbeg[i]} is the position
of the first entry of column \texttt{i} in both \texttt{matval} and
\texttt{matind}). By convention \texttt{matbeg} is allocated to be of length
\texttt{n+1}, with \texttt{matbeg[n]} containing the position after the very
last entry in \texttt{matval} and \texttt{matind} (so it is very conveniently
equal to \texttt{nz}). This representation of a matrix is known as a {\em
column ordered} or {\em column major} representation. Below are listed the
fields that must be filled out to describe a subproblem.\\
\\

\begin{description}
\item [\texttt{int n} --] The number of columns.
\item [\texttt{int m} --] The number of rows.
\item [\texttt{int nz} --] The number of nonzeros.
\item [\texttt{double obj\_offset} --] Constant to be added to the objective
function value when printed.
\item [\texttt{char obj\_sense} --] Objective sense (set to \texttt{MAXIMIZE}
or \texttt{MINIMIZE}).
\item [\texttt{char *is\_int} --] Indicates which variables are required to be 
integer.
\item [\texttt{int *matbeg} -- ] The array containing the starting positions
for each column.  
\item [\texttt{int *matind} --] The array containing the indices for each
column. 
\item [\texttt{double *matval} --] The array containing the matrix values for
each column.
\item [\texttt{double *obj} --] The objective function coefficients for the second objective (for multicriteria solve).
\item [\texttt{double *obj2} --] The objective function coefficients.
\item [\texttt{double *rhs} --] The right hand side values for the
constraints. 
\item [\texttt{double *rngval} --] The range values for the constraints
(optional). 
\item [\texttt{char *sense} --] The senses of the constraints. 
\item [\texttt{double *lb} --] The lower bounds of the variables.
\item [\texttt{double *ub} --] The upper bounds of the variables.
\item [\texttt{char **colname} --] The column names.
\end{description}

\begin{htmlonly}
\paragraph{cut\_data}
\end{htmlonly}
%\begin{latexonly}
\functiondef{cut\_data}
%\end{latexonly}
\label{cut_data}

Another of the internally defined data structures that the user has to
deal with frequently is the {\tt cut\_data} data structure, used to
store the packed form of cuts. This structure has 8 fields listed
below.
\begin{description}
\item [{\tt int size} --] The size of the {\tt coef} array.
\item [{\tt char *coef} --] An array containing the packed form of the
cut, which is defined and constructed by the user. Given this packed
form and a list of the variables active in the current relaxation, the
user must be able to construct the corresponding constraint.
\item [{\tt double rhs} --] The right hand side of the constraint.
\item [{\tt double range} --] The range of the constraint. It is zero for
a standard form constraint. Otherwise, the row activity level is
limited to between {\tt rhs} and ${\tt rhs}+{\tt range}$.
\item [{\tt char type} --] A user-defined type identifier that represents the
general class that the cut belongs to.
\item [{\tt char sense} --] The sense of the constraint. Can be either 'L' 
($\leq$), 'E' ($=$), 'G' ($\geq$) or 'R' (ranged). This may be evident
from the {\tt type}.
\item [{\tt char deletable} --] Determines whether or not a cut can be
deleted once added to the formulation. {\tt TRUE} by default.
\item [{\tt char branch} --] Determines whether the cut can be branched
on or not. Possible initial values are {\tt DO\_NOT\_BRANCH\_ON\_THIS\_ROW}
and {\tt ALLOWED\_TO\_BRANCH\_ON}.
\item [{\tt int name} --] Identifier used by \BB. The user should not
set this.
\end{description}

\begin{htmlonly}
\paragraph{waiting\_row}
\end{htmlonly}
%\begin{latexonly}
\functiondef{waiting\_row}
\label{waiting_row}
%\end{latexonly}

A closely related data structure is the {\tt waiting\_row},
essentially the ``unpacked'' form of a cut. There are six fields.

\begin{description}

\item[{\tt source\_pid} --] Used internally by \BB.

\item[{\tt cut\_data *cut} --] Pointer to the cut from which the row was generated.

\item[{\tt int nzcnt, *matind, *matval} --] Fields describing the row. {\tt nzcnt} is the number of nonzeros in
the row, i.e., the length of the {\tt matind} and {\tt matval} arrays,
which are the variable indices (wrt. the current LP relaxation) and
nonzero coefficients in the row.

\item[{\tt double violation} --] If the constraint corresponding to the cut is violated, this value
contains the degree of violation (the absolute value of the difference
between the row activity level (i.e., lhs) and the right hand
side). This value does not have to be set by the user.

\end{description}

\begin{htmlonly}
\paragraph{var\_desc}
\end{htmlonly}
%\begin{latexonly}
\functiondef{var\_desc}
%\end{latexonly}

The {\tt var\_desc} structure is used list the variables in the current
relaxation. There are four fields.

\begin{description}
\item[{\tt int userind} --] The user index of the variables,
\item[{\tt int colind} --] The column index of the variables (in the
current relaxation),
\item[{\tt double lb} --] The lower bound of the variable,
\item[{\tt double ub} --] The upper bound of the variable.
\end{description}

\end{description}

\newpage
\subsubsection{Function Descriptions}
\noindent Now we describe the functions themselves.

\bd

%%%%%%%%%%%%%%%%%%%%%%%%%%%%%%%%%%%%%%%%%%%%%%%%%%%%%%%%%%%%%%%%%%%%%%%%%%%%%
% user_receive_lp_data
%%%%%%%%%%%%%%%%%%%%%%%%%%%%%%%%%%%%%%%%%%%%%%%%%%%%%%%%%%%%%%%%%%%%%%%%%%%%%
\firstfuncdef{user\_receive\_lp\_data} 
\label{user_receive_lp_data}
\sindex[cf]{user\_receive\_lp\_data}
\begin{verbatim}
int user_receive_lp_data (void **user)
\end{verbatim}

\bd

\describe

This function only has to be filled out for parallel execution and only if
either the TM or LP modules are configured as separate processes. Otherwise,
data will have been copied into appropriate locations in the master function
\hyperref{{\tt user\_send\_lp\_data()}} {\ptt {user\_send\_lp\_data()} (see
Section }{)}{user_send_lp_data}. The two cases can be handled by means of {\tt
\#ifdef} statements. See comments in the source code stubs for more details. \\
\\
Here, the user must receive all problem-specific information sent from
the master, set up necessary data structures, etc. Note that the data must be
received in exactly the same order as it was sent in \hyperref{{\tt
user\_send\_lp\_data()}} {\ptt {user\_send\_lp\_data()} (see Section
}{)}{user_send_lp_data}. See Section \ref{communication} for more notes on
receiving data.

\args

\bt{llp{250pt}}
{\tt void **user} & OUT & Pointer to the user-defined LP data
structure. \\
\et

\returns

\bt{lp{300pt}}
{\tt USER\_ERROR} & Error. \BB\ aborts this LP module. \\
{\tt USER\_SUCCESS} & User received the data successfully. \\
{\tt USER\_DEFAULT} & User did not send any data. \\
\et

\item[Wrapper invoked from:] 
{\tt lp\_initialize()} at process start.

\ed

\vspace{1ex}

%%%%%%%%%%%%%%%%%%%%%%%%%%%%%%%%%%%%%%%%%%%%%%%%%%%%%%%%%%%%%%%%%%%%%%%%%%%%%
% user_create_subproblem
%%%%%%%%%%%%%%%%%%%%%%%%%%%%%%%%%%%%%%%%%%%%%%%%%%%%%%%%%%%%%%%%%%%%%%%%%%%%%

\functiondef{user\_create\_subproblem}
\label{user_create_subproblem}
\sindex[cf]{user\_create\_subproblem}
\begin{verbatim}
int user_create_subproblem(void *user, int *indices, MIPdesc *mip,
                           int *maxn, int *maxm, int *maxnz)
\end{verbatim}

\bd

\describe

Based on the instance data contained in the user data structure and the list
of base and extra variables that are active in the current subproblem, the
user has to create the subproblem for the search node. The matrix describing
the subproblem must contain those variables whose user indices are listed in
the array \texttt{indices} (in the same order) and the base constraints. The
extra (dynamically generated) constraints are added automatically by SYMPHONY
after the initial subproblem description is created. \\
\\
In this function, the user is required to construct a description of the
subproblem in column-ordered format and pass it back to SYMPHONY by filling
out the \texttt{MIPdesc} data structure, described \hyperref{here}{in Section
}{}{MIPdesc}. The user is not responsible for allocating extra memory to allow
for the addition of dynamically generated cuts and variables. The arrays
allocated in \ptt{user\_create\_subproblem()} are owned by SYMPHONY after
allocation and are freed as soon as the relaxation is loaded into the
solver. However, if the user has an idea as to the maximum number of variables
and constraints that are likely to be generated during processing of the
subproblem, this information can be passed to SYMPHONY in the variables
\texttt{*maxn}, \texttt{*maxm}, and \texttt{*maxnz}. These numbers are only
estimates that SYMPHONY can use to perform memory allocation. They do not have
to be exact numbers. In fact, these estimates need not be provided at all. The
\texttt{obj\_sense} and \texttt{obj\_offset} fields are set globally in the
function \hyperref{{\tt user\_initialize\_root\_node()}} {\ptt{
user\_initialize\_root\_node()} (see
Section}{)}{user_initialize_root_node}. Setting them here will have no
effect.\\
\\
Note that, the user should return ``USER\_DEFAULT'' if an MPS or GMPL/AMPL 
file was read in to describe the original MILP. \BB\ will allocate the 
corresponding arrays and specify the constraint matrix automatically in this 
case.

\args

\bt{llp{257pt}}
{\tt void *user} & IN & Pointer to the user-defined LP data structure. \\ & \\
{\tt int *indices} & IN & The list of the active variables (base and extra).\\
{\tt MIPdesc *mip} & OUT & Pointer to the MIPdesc data structure. \\ 
{\tt int *maxn} & OUT & Estimated maximum number of variables.\\
{\tt int *maxm} & OUT & Estimated maximum number of constraints.\\
{\tt int *maxnz} & OUT & Estimated maximum number of nonzeros. \\
\et

\returns

\bt{lp{330pt}}
{\tt USER\_ERROR} & Error. The LP module is aborted. \\
{\tt USER\_SUCCESS} & User created the constraint matrix with the base
constraints. \\
{\tt USER\_DEFAULT} & This return code is used when the default routines for
reading in an MPS or AMPL file were used and the user wants to let SYMPHONY
set up the subproblem automatically. This return will cause an error if the
default I/O routines were not used. \\
\et

\postp

The extra constraints are added to the matrix by calling the 
{\tt \htmlref{user\_unpack\_cuts()}{user_unpack_cuts}} subroutine and 
then adding the corresponding rows to the matrix. This is easier for
the user to implement, but less efficient than adding the cuts at the
time the original matrix was being constructed.

\item[Wrapper invoked from:] {\tt process\_chain()} which is
invoked when setting up a the initial search node in a chain.

\ed
\vspace{1ex}

%%%%%%%%%%%%%%%%%%%%%%%%%%%%%%%%%%%%%%%%%%%%%%%%%%%%%%%%%%%%%%%%%%%%%%%%%%%%%
% user_is_feasible
%%%%%%%%%%%%%%%%%%%%%%%%%%%%%%%%%%%%%%%%%%%%%%%%%%%%%%%%%%%%%%%%%%%%%%%%%%%%%

\functiondef{user\_is\_feasible}
\sindex[cf]{user\_is\_feasible}
\begin{verbatim}
int user_is_feasible(void *user, double lpetol, int varnum, int
                     *indices, double *values, int *feasible, 
                     double *objval, int branching, double *heur_solution)
\end{verbatim}

\bd
\describe

The user can test the feasibility status of the solution to the current LP
relaxation and/or return a feasible solution generated by a primal heuristic.
This function is primarily used in cases where a solution satisfying
integrality restrictions may not be feasible, i.e., it may violate an
inequality not present in the current relaxation that will be generated by the
user during the cut generation phase. In such a case, it is not possible for
SYMPHONY to determine this and the user should set {\tt *feasible} to {\tt
IP\_INFEASIBLE} and return {\tt USER\_SUCCESS}. If the user tests the
feasibility status of the solution and determines that the current solution is
feasible, then {\tt *feasible} should be set to {\tt IP\_FEASIBLE} instead. IN
this case, the user can set {\tt *objval} to the true objective function value
of the feasible solution to account for any round-ff error introduced by the
solver if desired. If all integral solution must be feasible, the user can ask
SYMPHONY to simply test the integrality status by returning {\tt
USER\_DEFAULT}.

If the user is able to generate a feasible solution using a primal heuristic,
then the solution can be returned in a dense format in the array {\tt
heur\_solution}. In this case, {\tt *objval} should be set to the value of the
solution and {\tt *feasible} should be set to {\tt IP\_HEUR\_FEASIBLE}. 

\args

\bt{llp{265pt}}
{\tt void *user} & INOUT & Pointer to the user-defined LP data structure. \\
& & \\
{\tt double lpetol} & IN & The $\epsilon$ tolerance of the LP solver. \\
{\tt int varnum} & IN & The length of the {\tt indices} and {\tt values}
arrays.\\ 
{\tt int *indices} & IN & User indices of variables at nonzero level in the
current solution.\\ 
{\tt double *values} & IN & Values of the variables listed in {\tt indices}.\\
& & \\
{\tt int *feasible} & OUT & Feasibility status of the solution ({\tt
IP\_INFEASIBLE}, {\tt IP\_FEASIBLE}, or {\tt IP\_HEUR\_FEASIBLE}). \\
{\tt double *objval} & INOUT & The user can return the ``true'' objective
function value of the solution in the case it is feasible, i.e., eliminating
the round-off error. This variable should also be set to the value of the
heuristic solution, if one is found. \\
{\tt int branching} & In & Indicates whether the function is being invoked
during strong branching (from {\tt select\_branching\_object()}) or after
solving an Lp relaxation (from {\tt fathom\_branch()}). \\
{\tt double *heur\_solution} & OUT & Used to return a solution determined
heuristically.
\et

\returns

\bt{lp{300pt}}
{\tt USER\_ERROR} & Error. Solution is considered to be not feasible.\\
{\tt USER\_SUCCESS} & User checked IP feasibility. \\
{\tt USER\_DEFAULT} & SYMPHONY should test the integrality of the solution to
determine if it is feasible.
\et

\item[Wrapper invoked from:] {\tt select\_branching\_object()} after
pre-solving the LP relaxation of a child corresponding to a candidate and from
{\tt fathom\_branch()} after solving an LP relaxation.

\ed
\vspace{1ex}

%%%%%%%%%%%%%%%%%%%%%%%%%%%%%%%%%%%%%%%%%%%%%%%%%%%%%%%%%%%%%%%%%%%%%%%%%%%%%
% user_send_feasible_solution
%%%%%%%%%%%%%%%%%%%%%%%%%%%%%%%%%%%%%%%%%%%%%%%%%%%%%%%%%%%%%%%%%%%%%%%%%%%%%

\functiondef{user\_send\_feasible\_solution}
\label{user_send_feasible_solution}
\sindex[cf]{user\_send\_feasible\_solution}
\begin{verbatim}
int user_send_feasible_solution(void *user, double lpetol,
                                int varnum, int *indices, double *values)
\end{verbatim}

\bd
\describe

This function is only used for parallel computation. The user can send a
feasible solution in custom format to the master module if desired. However,
the default routine suffices in most situations. The solution is sent using
the communication functions described in Section
\ref{communication} in whatever logical format the user wants to use.
The default is to pack the user indices and values of variables at
non-zero level. If the user packs the
solution herself then the same data must be packed here that will be
received in the {\tt \htmlref{user\_receive\_feasible\_solution()}
{user_receive_feasible_solution}} function in
the master module. See the description of that function for details.
This function will only be called when either the LP or tree manager
are running as a separate executable. Otherwise, the solution gets
stored within the LP user data structure.

\args

\bt{llp{290pt}}
{\tt void *user} & IN & Pointer to the user-defined LP data structure. \\
& & \\
{\tt double lpetol} & IN & The $\epsilon$ tolerance of the LP solver. \\
{\tt int varnum} & IN & The length of the {\tt indices} and {\tt
values} arrays.\\
{\tt int *indices} & IN & User indices of variables at nonzero level
in the current solution.\\
{\tt double *values} & IN & Values of the variables listed in {\tt indices}.\\
\et

\returns

\bt{lp{315pt}}
{\tt USER\_ERROR} & Error. Do the default.\\
{\tt USER\_SUCCESS} & User packed the solution.\\
{\tt USER\_DEFAULT} & Regulated by the parameter {\tt
pack\_feasible\_solution\_default}, but set to {\tt SEND\_NONZEROS}
unless overridden by the user.\\
{\tt SEND\_NONZEROS} & Pack the nonzero values and their indices.\\
\et

\item[Wrapper invoked:] as soon as feasibility is detected anywhere.

\ed
\vspace{1ex}

%%%%%%%%%%%%%%%%%%%%%%%%%%%%%%%%%%%%%%%%%%%%%%%%%%%%%%%%%%%%%%%%%%%%%%%%%%%%%
% user_display_lp_solution
%%%%%%%%%%%%%%%%%%%%%%%%%%%%%%%%%%%%%%%%%%%%%%%%%%%%%%%%%%%%%%%%%%%%%%%%%%%%%

\functiondef{user\_display\_lp\_solution}
\sindex[cf]{user\_display\_lp\_solution}
\begin{verbatim}
int user_display_lp_solution(void *user, int which_sol,
                          int varnum, int *indices, double *values)
\end{verbatim}

\bd
\describe

Given a solution to an LP relaxation (the indices and values of the nonzero
variables) the user can (graphically) display it. The {\tt which\_sol}
argument shows what kind of solution is passed to the function: {\tt
DISP\_FEAS\_SOLUTION} indicates a solution feasible to the original IP
problem, {\tt DISP\_RELAXED\_SOLUTION} indicates the solution to any LP
relaxation and {\tt DISP\_FINAL\_RELAXED\_SOLUTION} indicates the solution to
an LP relaxation when no cut has been found. There is no post-processing.
Default options print out user indices and values of nonzero or fractional
variables on the standard output.

\args

\bt{llp{290pt}}
{\tt void *user} & IN & Pointer to the user-defined LP data structure.\\
& & \\
{\tt int which\_sol} & IN & The type of solution passed on to the
displaying function. Possible values are {\tt DISP\_FEAS\_SOLUTION},
{\tt DISP\_RELAXED\_SOLUTION} and {\tt DISP\_FINAL\_RELAXED\_SOLUTION}. \\
{\tt int varnum} & IN & The number of variables in the current
solution at nonzero level (the length of the {\tt indices} and {\tt
values} arrays). \\
{\tt int *indices} & IN & User indices of variables at
nonzero level in the current solution.\\
{\tt double *values} & IN & Values of the nonzero variables.\\
\et

\returns

\bt{lp{317pt}}
{\tt USER\_ERROR} & Error. \BB\ ignores error message. \\
{\tt USER\_SUCCESS} & User displayed whatever she wanted to.\\
{\tt USER\_DEFAULT} & Regulated by the parameter {\tt
display\_solution\_default}, but set to {\tt DISP\_NZ\_INT} unless overridden
by the user.\\
{\tt DISP\_NOTHING} & Display nothing. \\
{\tt DISP\_NZ\_INT} & Display user indices (as integers) and values of
nonzero variables. \\
{\tt DISP\_NZ\_HEXA} & Display user indices (as hexadecimals) and
values of nonzero variables. \\
{\tt DISP\_FRAC\_INT} & Display user indices (as integers) and values
of variables not at their lower or upper bounds. \\
{\tt DISP\_FRAC\_HEXA} & Display user indices (as hexadecimals) and
values of variables not at their lower and upper bounds. \\
\et

\item[Wrapper invoked from:] {\tt fathom\_branch()} with {\tt
DISP\_FEAS\_SOLUTION} or {\tt DISP\_RELAXED\_SOLUTION} after solving an LP
relaxation and checking its feasibility status. If it was not feasible and no
cut could be added either then the wrapper is invoked once more, now with {\tt
DISP\_FINAL\_RELAXED\_SOLUTION}. 

\ed
\vspace{1ex}

%%%%%%%%%%%%%%%%%%%%%%%%%%%%%%%%%%%%%%%%%%%%%%%%%%%%%%%%%%%%%%%%%%%%%%%%%%%%%
% user_shall_we_branch  and  user_select_candidates
%%%%%%%%%%%%%%%%%%%%%%%%%%%%%%%%%%%%%%%%%%%%%%%%%%%%%%%%%%%%%%%%%%%%%%%%%%%%%

\functiondef{user\_shall\_we\_branch}
\label{user_shall_we_branch}
\sindex[cf]{user\_shall\_we\_branch}
\begin{verbatim}
int user_shall_we_branch(void *user, double lpetol, int cutnum, 
                         int slacks_in_matrix_num,
                         cut_data **slacks_in_matrix,
                         int slack_cut_num, cut_data **slack_cuts,
                         int varnum, var_desc **vars, double *x, 
                         char *status, int *cand_num, 
                         branch_obj ***candidates, int *action)
\end{verbatim}

\bd

\describe

There are two user-written functions invoked from {\tt
select\_candidates\_u}. The first one
({\tt \htmlref{user\_shall\_we\_branch()}{user_shall_we_branch}}) decides
whether to branch at all, the second one ({\tt
\htmlref{user\_select\_candidates()}{user_select_candidates}}) chooses
the branching objects. The argument lists of the two functions are the
same, and if branching occurs (see discussion below) then the contents
of {\tt *cand\_num} and {\tt *candidates} will not change between the
calls to the two functions.\\
\\
The first of these two functions is invoked in each iteration
after solving the LP relaxation and (possibly) generating cuts.
Therefore, by the time it is called, some violated cuts might be known.
Still, the user might decide to branch anyway. The second function is
invoked only when branching is decided on.\\
\\
Given (1) the number of known violated cuts that can be added to the
problem when this function is invoked, (2) the constraints that are
slack in the LP relaxation, (3) the slack cuts not in the matrix that
could be branched on (more on this later), and (4) the solution to the
current LP relaxation, the user must decide whether to branch or not.
Branching can be done either on variables or slack cuts. A pool of
slack cuts which has been removed from the problem and kept for
possible branching is passed to the user. If any of these happen to
actually be violated (it is up to the user to determine this), they
can be passed back as branching candidate type {\tt VIOLATED\_SLACK}
and will be added into the current relaxation. In this case, branching
does not have to occur (the structure of the {\tt *candidates} array
is described below in {\tt \htmlref{user\_select\_candidates()}
{user_select_candidates}}). \\
\\
This function has two outputs. The first output is {\tt *action} which
can take four values: {\tt USER\_\_DO\_BRANCH} if the user wants to
branch, {\tt USER\_\_DO\_NOT\_BRANCH} if he doesn't want to branch,
{\tt USER\_\_BRANCH\_IF\_MUST} if he wants to branch only if there are
no known violated cuts, or finally {\tt USER\_\_BRANCH\_IF\_TAILOFF} if he
wants to branch in case tailing off is detected. The second output is the
number of candidates and their description. In this function the only
sensible ``candidates'' are {\tt VIOLATED\_SLACK}s. \\
\\
There is no post processing, but in case branching is
selected, the {\tt col\_gen\_before\_branch()} function is invoked
before the branching would take place. If that function finds dual
infeasible variables then (instead of branching) they are added to the
LP relaxation and the problem is resolved. (Note that the behavior of
the {\tt col\_gen\_before\_branch()} is governed by the {\tt
colgen\_strat[]} TM parameters.)

\args

\bt{llp{205pt}}
{\tt void *user} &  IN & Pointer to the user-defined LP data structure. \\
{\tt double lpetol} & IN & The $\epsilon$ tolerance of the LP solver. \\
& & \\
{\tt int cutnum} & IN & The number of violated cuts (known before invoking
this function) that could be added to the problem (instead of branching).\\
& & \\
{\tt int slacks\_in\_matrix\_num} & IN & Number of slack constraints in the
matrix. \\
{\tt cut\_data **slacks\_in\_matrix} & IN & The description of the cuts
corresponding to these constraints (see Section \ref{cut_data}). \\
& & \\
{\tt int slack\_cut\_num} & IN & The number of slack cuts not in the
matrix. \\
{\tt cut\_data **slack\_cuts} & IN & Array of pointers to these cuts
(see Section \ref{cut_data}).\\
{\tt int varnum} & IN & The number of variables in the current lp
relaxation (the length of the following three arrays).\\
{\tt var\_desc **vars} & IN & Description of the variables in the
relaxation. \\
{\tt double *x} & IN & The corresponding solution values (in the optimal
solution to the relaxation).\\
{\tt char *status} & IN & The stati of the variables. 
There are five possible status values: 
{\tt NOT\_\-FIXED}, {\tt TEMP\_\-FIXED\_\-TO\_\-UB}, {\tt
PERM\_\-FIXED\_\-TO\_\-UB}, {\tt TEMP\_\-FIXED\_\-TO\_\-LB} and {\tt
PERM\_\-FIXED\_\-TO\_\-LB}.\\ 
& & \\
{\tt int *cand\_num} & OUT & Pointer to the number of candidates
returned (the length of {\tt *candidates}).\\
{\tt candidate ***candidates} & OUT & Pointer to the array of
candidates generated (see description below).\\
{\tt int *action} & OUT & What to do. Must be one of the four above described
values unless the return code is {\tt USER\_DEFAULT}.
\et

\returns

\bt{lp{340pt}}
{\tt USER\_ERROR} & Error. {\tt DEFAULT} is used. \\
{\tt USER\_SUCCESS} & The user filled out {\tt *action} (and possibly {\tt
*cand\_num} and {\tt *candidates}). \\
{\tt USER\_DEFAULT} & Action taken is controlled by the parameter {\tt
shall\_we\_branch\_default}, which is initially
{\tt USER\_\_BRANCH\_IF\_MUST} unless overridden by the user. \\ 
\et

\item[Notes:] \hfill
\begin{itemize}
	\item The user has to allocate the pointer array for the
	candidates and place the pointer for the array into {\tt
	***candidates} (if candidates are returned).

	\item Candidates of type {\tt VIOLATED\_SLACK} are always
	added to the LP relaxation regardless of what {\tt action} is
	chosen and whether branching will be carried out or not.

	\item Also note that the user can change his mind in 
	{\tt \htmlref{user\_select\_candidates()}
	{user_select_candidates}} and not branch after all, even if
	she chose to branch in this function. A possible scenario:
	{\tt cut\_num} is zero when this function is invoked and the
	user asks for {\tt USER\_\_BRANCH\_IF\_MUST} without checking
	the slack constraints and slack cuts. Afterward no columns
	are generated (no dual infeasible variables found) and thus
	\BB\ decides branching is called for and invokes 
	{\tt \htmlref{user\_select\_candidates()}{user_select_candidates}}. 
	However, in that function the
	user checks the slack cuts, finds that some are violated,
	cancels the branching request and adds the violated cuts to
	the relaxation instead.
\end{itemize}

\item [\bf Warning:] 

The cuts the user unpacks and wants to be added to the
problem (either because they are of type {\tt VIOLATED\_SLACK} or type
{\tt CANDIDATE\_CUT\_NOT\_IN\_MATRIX}) will be deleted from the list
of slack cuts after this routine returns. Therefore the same
warning applies here as in the function {\tt 
\htmlref{user\_unpack\_cuts()}{user_unpack_cuts}}.

\item[Wrapper invoked from:] {\tt select\_branching\_object()}.

\ed

%%%%%%%%%%%%%%%%%%%%%%%%%%%%%%%%%%%%%%%%%%%%%%%%%%%%%%%%%%%%%%%%%%%%%%%%%%%%%
% user_select_candidates
%%%%%%%%%%%%%%%%%%%%%%%%%%%%%%%%%%%%%%%%%%%%%%%%%%%%%%%%%%%%%%%%%%%%%%%%%%%%%
\functiondef{user\_select\_candidates}
\label{user_select_candidates}
\sindex[cf]{user\_select\_candidates}
\begin{verbatim}
int user_select_candidates(void *user, double lpetol, int cutnum, 
                           int slacks_in_matrix_num,
                           cut_data **slacks_in_matrix,
                           int slack_cut_num, cut_data **slack_cuts,
                           int varnum, var_desc **vars, double *x, 
                           char *status, int *cand_num, 
                           branch_obj ***candidates, int *action,
                           int bc_level)
\end{verbatim}

\bd

\describe

The purpose of this function is to generate branching candidates. Note
that {\tt *action} from {\tt \htmlref{user\_shall\_we\_branch()}
{user_shall_we_branch}} is passed on to
this function (but its value can be changed here, see notes at the
previous function), as well as the candidates in {\tt **candidates}
and their number in {\tt *cand\_num} if there were any. \\
\\
Violated cuts found among the slack cuts (not in the matrix) can be
added to the candidate list. These violated cuts will be added to the
LP relaxation regardless of the value of {\tt *action}. \\
\\
The {\tt branch\_obj} structure contains fields similar to the
{\tt \htmlref{cut\_data}{cut_data}} data structure. Branching is
accomplished by imposing inequalities which divide the current
subproblem while cutting off the corresponding fractional solution.
Branching on cuts and variables is treated symmetrically and branching
on a variable can be thought of as imposing a constraint with a single
unit entry in the appropriate column. Following is a list of the
fields of the {\tt branch\_obj} data structure which must be set by
the user.

\begin{description}

\item[\tt char type] Can take five values:
\begin{description}
\item[\tt CANDIDATE\_VARIABLE] The object is a variable.
\item[\tt CANDIDATE\_CUT\_IN\_MATRIX] The object is a cut (it must
be slack) which is in the current formulation.
\item[\tt CANDIDATE\_CUT\_NOT\_IN\_MATRIX] The object is a cut (it
must be slack) which has been deleted from the formulation and is listed among
the slack cuts.
\item[\tt VIOLATED\_SLACK] The object is not offered as a candidate for
branching, but rather it is selected because it was among the slack cuts but
became violated again.
\item[\tt SLACK\_TO\_BE\_DISCARDED] The object is not selected as a candidate
for branching rather it is selected because it is a slack cut which should be
discarded even from the list of slack cuts.
\end{description}

\item[\tt int position]

The position of the object in the appropriate array (which is one of {\tt
vars}, {\tt slacks\_in\_matrix}, or {\tt slack\_cuts}.

\item[\tt waiting\_row *row] 

Used only if the type is {\tt CANDIDATE\_CUT\_NOT\_IN\_MATRIX} or {\tt
VIOLATED\_SLACK}. In these cases this field holds the row extension
corresponding to the cut. This structure can be filled out easily
using a call to {\tt \htmlref{user\_unpack\_cuts()}{user_unpack_cuts}}.

\item[\tt int child\_num] \hfill

The number of children of this branching object.

\item[\tt char *sense, double *rhs, double *range, int *branch] \hfill

The description of the children. These arrays determine the sense,
rhs, etc. for the cut to be imposed in each of the children. These are
defined and used exactly as in the {\tt \htmlref{cut\_data}{cut_data}}
data structure. {\bf Note:} If a limit is defined on the number of
children by defining the {\tt MAX\_CHILDREN\_NUM} macro to be a number
(it is pre-defined to be 4 as a default), then these arrays will be
statically defined to be the correct length and don't have to be
allocated. This option is highly recommended. Otherwise, the user must
allocate them to be of length {\tt child\_num}.

\item[\tt double lhs] 

The activity level for the row (for branching cuts). This field is
purely for the user's convenience. \BB\ doesn't use it so it need not
be filled out.

\item[\tt double *objval, int *termcode, int *iterd, int *feasible] \hfill

The objective values, termination codes, number of iterations and
feasibility stati of the children after pre-solving them. These are all
filed out by \BB\ during strong branching. The user may access them in
{\tt \htmlref{user\_compare\_candidates()}{user_compare_candidates}} 
(see below).

\end{description}

There are three default options (see below), each chooses a few variables (the
number is determined by the \hyperref{strong branching parameters}
{strong branching parameters (see Section } {)} {strong_branching}.

\args

Same as for \htmlref{\tt user\_shall\_we\_branch()}
{user_shall_we_branch}, except
that {\tt *action} must be either {\tt USER\_\_DO\_BRANCH} or {\tt
USER\_\_DO\_NOT\_BRANCH}, and if branching is asked for, there must be
a real candidate in the candidate list (not only {\tt
VIOLATED\_SLACK}s and {\tt SLACK\_TO\_BE\_DISCARDED}s). Also, the
argument {\tt bc\_level} is the level in the tree. This could be used
in deciding how many strong branching candidates to use.

\returns

\bt{lp{210pt}}
{\tt USER\_ERROR} & Error. {\tt DEFAULT} is used. \\
{\tt USER\_SUCCESS} & User generated branching candidates. \\
{\tt USER\_DEFAULT} & Regulated by the {\tt select\_candidates\_default} 
parameter, but set to {\tt USER\_\_CLOSE\_TO\_HALF} unless overridden by the
user.\\
{\tt USER\_\_CLOSE\_TO\_HALF} & Choose variables with values closest to half.\\
{\tt USER\_\_CLOSE\_TO\_HALF\_AND\_EXPENSIVE} & Choose variables
with values close to half and with high objective function coefficients.\\
{\tt USER\_\_CLOSE\_TO\_ONE\_AND\_CHEAP} & Choose variables with
values close to one and with low objective function coefficients.\\
\et

\item[Wrapper invoked from:] {\tt select\_branching\_object()}.

\item[Notes:] See the notes at {\tt \htmlref{user\_shall\_we\_branch()}
{user_shall_we_branch}}.

\ed

\vspace{1ex}


%%%%%%%%%%%%%%%%%%%%%%%%%%%%%%%%%%%%%%%%%%%%%%%%%%%%%%%%%%%%%%%%%%%%%%%%%%%%%
% user_compare_candidates
%%%%%%%%%%%%%%%%%%%%%%%%%%%%%%%%%%%%%%%%%%%%%%%%%%%%%%%%%%%%%%%%%%%%%%%%%%%%%

\functiondef{user\_compare\_candidates}
\label{user_compare_candidates}
\sindex[cf]{user\_compare\_candidates}
\begin{verbatim}
int user_compare_candidates(void *user, branch_obj *can1, branch_obj *can2,
                            double ub, double granularity, 
                            int *which_is_better)
\end{verbatim}

\bd
\describe

By the time this function is invoked, the children of the current search tree
node corresponding to each branching candidate have been pre-solved, i.e., the
{\tt objval}, {\tt termcode}, {\tt iterd}, and {\tt feasible} fields of the
{\tt can1} and {\tt can2} structures are filled out. Note that if the
termination code for a child is {\tt LP\_D\_UNBOUNDED} or {\tt LP\_D\_OBJLIM},
i.e., the dual problem is unbounded or the objective limit is reached, then
the objective value of that child is set to {\tt MAXDOUBLE / 2}. Similarly, if
the termination code is one of {\tt LP\_D\_ITLIM} (iteration limit reached),
{\tt LP\_D\_INFEASIBLE} (dual infeasible) or {\tt LP\_ABANDONED} (because of
numerical difficulties) then the objective value of that child is set to that
of the parent's.\\
\\
Based on this information the user must choose which candidate he
considers better and whether to branch on this better one immediately
without checking the remaining candidates. As such, there are four
possible answers: {\tt FIRST\_CANDIDATE\_BETTER}, \hfill {\tt
SECOND\_CANDIDATE\_BETTER}, \hfill {\tt
FIRST\_CANDIDATE\_BETTER\_AND\_BRANCH\_ON\_IT} and {\tt
SECOND\_CANDIDATE\_BETTER\_AND\_BRANCH\_ON\_IT}. An answer ending with
{\tt \_AND\_BRANCH\_ON\_IT} indicates that the user wants to terminate
the strong branching process and select that particular candidate for
branching.\\
\\
There are several default options. In each of them, objective values of
the pre-solved LP relaxations are compared. 

\args

\bt{llp{245pt}}
{\tt void *user} &  IN & Pointer to the user-defined LP data structure. \\
& & \\
{\tt branch\_obj *can1} & IN & One of the candidates to be compared.\\
{\tt branch\_obj *can2} & IN & The other candidate to be compared. \\
{\tt double ub} & IN & The current best upper bound. \\
{\tt double granularity} & IN & Defined tolerance  \\
{\tt int *which\_is\_better} & OUT & The user's choice. See the description
above. \\
\et

\newpage
\returns

\bt{lp{290pt}}
{\tt USER\_ERROR} & Error. {\tt DEFAULT} is used. \\
{\tt USER\_SUCCESS} & User filled out {\tt *which\_is\_better}. \\
{\tt USER\_DEFAULT} & Regulated by the {\tt compare\_candidates\_default}
parameter, initially set to {\tt LOWEST\_LOW\_OBJ} unless overridden by the
user. \\
{\tt BIGGEST\_DIFFERENCE\_OBJ} & Prefer the candidate with the biggest
difference between highest and lowest objective function values.\\
{\tt LOWEST\_LOW\_OBJ} & Prefer the candidate with the lowest minimum
objective function value. The minimum is taken over the objective function
values of all the children. \\
{\tt HIGHEST\_LOW\_OBJ} & Prefer the candidate with the highest minimum
objective function value. \\
{\tt LOWEST\_HIGH\_OBJ} & Prefer the candidate with the lowest maximum
objective function value. \\
{\tt HIGHEST\_HIGH\_OBJ} & Prefer the candidate with the highest maximum
objective function value .\\
{\tt LOWEST\_LOW\_FRAC} & Prefer the candidate with the lowest minimum
number of fractional variables. The minimum is taken over the number of
fractional variables in all the children. Fractional branching options are 
only available if the fractional branching compile-time option is set in the
makefile.\\ 
{\tt HIGHEST\_LOW\_FRAC} & Prefer the candidate with the highest minimum
number of fractional variables. \\
{\tt LOWEST\_HIGH\_FRAC} & Prefer the candidate with the lowest maximum
number of fractional variables. \\
{\tt HIGHEST\_HIGH\_FRAC} & Prefer the candidate with the highest maximum
number of fractional variables. \\
\et

\item[Wrapper invoked from:] {\tt select\_branching\_object()} after
the LP relaxations of the children have been pre-solved.

\ed

\vspace{1ex}


%%%%%%%%%%%%%%%%%%%%%%%%%%%%%%%%%%%%%%%%%%%%%%%%%%%%%%%%%%%%%%%%%%%%%%%%%%%%%
% user_select_child
%%%%%%%%%%%%%%%%%%%%%%%%%%%%%%%%%%%%%%%%%%%%%%%%%%%%%%%%%%%%%%%%%%%%%%%%%%%%%

\functiondef{user\_select\_child}
\label{user_select_child}
\sindex[cf]{user\_select\_child}
\begin{verbatim}
int user_select_child(void *user, double ub, branch_obj *can, char *action)
\end{verbatim}

\bd

\describe

By the time this function is invoked, the candidate for branching has
been chosen. Based on this information and the current best upper
bound, the user has to decide what to do with each child. Possible
actions for a child are {\tt KEEP\_THIS\_CHILD} (the child will be
kept at this \LP\ for further processing, i.e., the process {\em
dives} into that child), {\tt PRUNE\_THIS\_CHILD} (the child will be
pruned based on some problem specific property---no questions
asked...), {\tt PRUNE\_THIS\_CHILD\_FATHOMABLE} (the child will be
pruned based on its pre-solved LP relaxation) and {\tt
RETURN\_THIS\_CHILD} (the child will be sent back to tree manager).
Note that at most one child can be kept at the current LP module.

There are two default options---in both of them, objective values of the
pre-solved LP relaxations are compared (for those children whose pre-solve
did not terminate with primal infeasibility or high cost). One rule
prefers the child with the lowest objective function value and the
other prefers the child with the higher objective function value.

\args

\bt{llp{270pt}}
{\tt void *user} &  IN & Pointer to the user-defined LP data structure. \\
& & \\
{\tt int ub} & IN & The current best upper bound. \\
{\tt branch\_obj *can} & IN & The branching candidate. \\
& & \\
{\tt char *action} & OUT & Array of actions for the children. The array is
already allocated to length {\tt can->number}.\\
\et

\returns

\bt{lp{265pt}}
{\tt USER\_ERROR} & Error. {\tt DEFAULT} is used. \\
{\tt USER\_SUCCESS} & User filled out {\tt *action}. \\
{\tt USER\_DEFAULT} & Regulated by the {\tt select\_child\_default}
parameter, which is initially set to {\tt PREFER\_LOWER\_OBJ\_VALUE},
unless overridden by the user. \\
{\tt PREFER\_HIGHER\_OBJ\_VALUE} & Choose child with the highest
objective value.\\
{\tt PREFER\_LOWER\_OBJ\_VALUE} & Choose child with the lowest
objective value.\\
{\tt PREFER\_MORE\_FRACTIONAL} & Choose child with the most fractional
variables. Fractional branching options are 
only available if the fractional branching compile-time option is set in the
makefile.\\
{\tt PREFER\_LESS\_FRACTIONAL} & Choose child with the lowest number of
fractional variables.\\
\et

\postp

Checks which children can be fathomed based on the
objective value of their pre-solved LP relaxation.

\item[Wrapper invoked from:] {\tt branch()}.

\ed

\vspace{1ex}

%%%%%%%%%%%%%%%%%%%%%%%%%%%%%%%%%%%%%%%%%%%%%%%%%%%%%%%%%%%%%%%%%%%%%%%%%%%%%
% user_print_branch_stat
%%%%%%%%%%%%%%%%%%%%%%%%%%%%%%%%%%%%%%%%%%%%%%%%%%%%%%%%%%%%%%%%%%%%%%%%%%%%%

\functiondef{user\_print\_branch\_stat}
\sindex[cf]{user\_print\_branch\_stat}
\begin{verbatim}
int user_print_branch_stat(void *user, branch_obj *can, cut_data *cut,
                           int n, var_desc **vars, char *action)
\end{verbatim}

\bd

\describe

Print out information about branching candidate {\tt can}, such as a
more explicit problem-specific description than \BB\ can provide (for
instance, end points of an edge). If {\tt verbosity} is set
high enough, the identity of the branching object and the children
(with objective values and termination codes for the pre-solved LPs) is
printed out to the standard output by \BB.

\args

\bt{llp{285pt}}
{\tt void *user} &  IN & Pointer to the user-defined LP data structure. \\
& & \\
{\tt branch\_obj *can} & IN & The branching candidate. \\
{\tt cut\_data *cut} & IN & The description of the cut if the
branching object is a cut. \\
{\tt int n} & IN & Number of variables. \\
{\tt var\_desc **vars} & IN & Array of variables in the current relaxation.\\
{\tt char *action} & IN & Array of actions for the children. \\
\et

\returns

\bt{lp{300pt}}
{\tt USER\_ERROR} & Error. Ignored by \BB. \\
{\tt USER\_SUCCESS} & The user printed out whatever she wanted to. \\
{\tt USER\_DEFAULT} & SYMPHONY prints out its own branching information. \\
\et

\item[Wrapper invoked from:] {\tt branch()} after the best candidate has been
selected, pre-solved, and the action is decided on for the children.

\ed

\vspace{1ex}


%%%%%%%%%%%%%%%%%%%%%%%%%%%%%%%%%%%%%%%%%%%%%%%%%%%%%%%%%%%%%%%%%%%%%%%%%%%%%
% user_add_to_desc
%%%%%%%%%%%%%%%%%%%%%%%%%%%%%%%%%%%%%%%%%%%%%%%%%%%%%%%%%%%%%%%%%%%%%%%%%%%%%

\functiondef{user\_add\_to\_desc}
\sindex[cf]{user\_add\_to\_desc}
\begin{verbatim}
int user_add_to_desc(void *user, int *desc_size, char **desc)
\end{verbatim}

\bd
\describe

Before a node description is sent to the TM, the user can provide a
pointer to a data structure that will be appended to the description
for later use by the user in reconstruction of the node. This
information must be placed into {\tt *desc}. Its size should be
returned in {\tt *desc\_size}.\\
\\
There is only one default option: the description to be added is considered to
be of zero length, i.e., there is no additional description.

\args

\bt{llp{275pt}}
{\tt void *user} & IN & Pointer to the user-defined LP data structure. \\
& & \\
{\tt int *desc\_size} & OUT & The size of the additional information, the
length of {\tt *desc} in bytes. \\
{\tt char **desc} & OUT & Pointer to the additional information (space must be
allocated by the user).\\
\et

\returns

\bt{lp{300pt}}
{\tt USER\_ERROR} & Error. {\tt DEFAULT} is used. \\
{\tt USER\_SUCCESS} & User filled out {\tt *desc\_size} and {\tt *desc}.\\
{\tt USER\_DEFAULT} & No description is appended. \\
\et

\item[Wrapper invoked from:] {\tt create\_explicit\_node\_desc()} before a
node is sent to the tree manager.

\ed
\vspace{1ex}

%%%%%%%%%%%%%%%%%%%%%%%%%%%%%%%%%%%%%%%%%%%%%%%%%%%%%%%%%%%%%%%%%%%%%%%%%%%%%
% user_same_cuts
%%%%%%%%%%%%%%%%%%%%%%%%%%%%%%%%%%%%%%%%%%%%%%%%%%%%%%%%%%%%%%%%%%%%%%%%%%%%%

\functiondef{user\_same\_cuts}
\sindex[cf]{user\_same\_cuts}
\begin{verbatim}
int user_same_cuts (void *user, cut_data *cut1, cut_data *cut2, 
                    int *same_cuts)
\end{verbatim}

\bd
\describe

Determine whether the two cuts are comparable (the normals of the half-spaces
corresponding to the cuts point in the same direction) and if yes, which one
is stronger. The default is to declare the cuts comparable only if the {\tt
type}, {\tt sense} and {\tt coef} fields of the two cuts are the same byte by
byte; and if this is the case to compare the right hand sides to decide which
cut is stronger.

\args

\bt{llp{275pt}}
{\tt void *user} & IN & Pointer to the user-defined LP data structure.\\
& & \\
{\tt cut\_data *cut1} & IN & The first cut.\\
{\tt cut\_data *cut2} & IN & The second cut.\\
{\tt int *same\_cuts} & OUT & Possible values: {\tt SAME}, {\tt
FIRST\_CUT\_BETTER}, {\tt SECOND\_CUT\_BETTER} and {\tt DIFFERENT} (i.e., not
comparable).\\
\et

\returns

\bt{lp{295.386pt}}
{\tt USER\_ERROR} & Error. {\tt DEFAULT} is used.\\
{\tt USER\_SUCCESS} & User did the comparison, filled out {\tt *same\_cuts}. \\
{\tt USER\_DEFAULT} & Compare byte by byte (see above). \\
\et

\item[Wrapper invoked from:] {\tt process\_message()} when a {\tt PACKED\_CUT}
arrives.

\item[Note:] \hfill

This function is used to check whether a newly arrived cut is already
in the local pool. If so, or if it is weaker than a cut in the local
pool, then the new cut is discarded; if it is stronger then a cut in
the local pool, then the new cut replaces the old one and if the new
is different from all the old ones, then it is added to the local pool.

\ed

%%%%%%%%%%%%%%%%%%%%%%%%%%%%%%%%%%%%%%%%%%%%%%%%%%%%%%%%%%%%%%%%%%%%%%%%%%%%%
% user_unpack_cuts
%%%%%%%%%%%%%%%%%%%%%%%%%%%%%%%%%%%%%%%%%%%%%%%%%%%%%%%%%%%%%%%%%%%%%%%%%%%%%

\functiondef{user\_unpack\_cuts}
\label{user_unpack_cuts}
\sindex[cf]{user\_unpack\_cuts}
\begin{verbatim}
int user_unpack_cuts(void *user, int from, int type, int varnum, 
                     var_desc **vars, int cutnum, cut_data **cuts,
                     int *new_row_num, waiting_row ***new_rows)
\end{verbatim}

\bd
\describe

If the user has defined application-specific cut classes, these cuts must be
interpreted as constraints for the current LP relaxation, i.e., the user must
decode the compact representation of the cuts (see the {\tt
\htmlref{cut\_data}{cut_data}} structure) into rows for the matrix. A pointer
to the array of generated rows must be returned in {\tt ***new\_rows} (the
user has to allocate this array) and their number in {\tt *new\_row\_num}. \\
\\
Note that SYMPHONY has built-in support for cuts generated explicitly as
matrix rows with no user-defined packed form, i.e., cuts whose indices and
coefficients are given explicitly (see the function
\hyperref{{\tt user\_find\_cuts()}} {\ptt {user\_find\_cuts()} in Section
}{}{user_find_cuts}. These cuts can be constructed and added using the helper
function
\hyperref{\texttt{cg\_add\_explicit\_cut()}} {\texttt{cg\_add\_explicit\_cut()} (see
the description of \ptt{user\_find\_cuts()} in Section }{)}{user_find_cuts}
and are packed and unpacked automatically, so the user does not need to
implement this function. In post processing, SYMPHONY unpacks explicitly
defined cuts and internally generated generic cuts.

\args

\bt{llp{225pt}}
{\tt void *user} & IN & Pointer to the user-defined LP data structure. \\
& & \\
{\tt int from} & IN & See below in ``Notes''. \\
{\tt int type} & IN & {\tt UNPACK\_CUTS\_SINGLE} or {\tt
UNPACK\_CUTS\_MULTIPLE} (see notes below). \\
{\tt int varnum} & IN & The number of variables. \\
{\tt var\_desc **vars} & IN & The variables currently in the problem. \\
{\tt int cutnum} & IN & The number of cuts to be decoded. \\
{\tt cut\_data **cuts} & IN & Cuts that need to be converted to rows
for the current LP. See ``Warning'' below. \\
& & \\
{\tt int *new\_row\_num} & OUT & Pointer to the number of rows in
{\tt **new\_rows}. \\
{\tt waiting\_row ***new\_rows} & OUT & Pointer to the array of
pointers to the new rows. \\
\et

\returns

\bt{lp{300pt}}
{\tt USER\_ERROR} & Error. The cuts are discarded.\\
{\tt USER\_SUCCESS} & User unpacked the cuts. \\
{\tt USER\_DEFAULT} & There are no user cut types defined. In this case,
SYMPHONY deals with only explicit cuts and internally generated cuts.  \\
\et

\item[Wrapper invoked from:] Wherever a cut needs to be unpacked
(multiple places).

\postp

Explicit row cuts are processed, as well as SYMPHONY's internally generated
cuts. Also, the pointers to each cut are transferred to the
\texttt{waiting\_rows} data structure (in previous version, this was done by
the user).   

\newpage
\item[Notes:] \hfill

\begin{itemize}

\item When decoding the cuts, the expanded constraints have to be adjusted to
the current LP, i.e., coefficients corresponding to variables currently not in
the LP have to be left out.

\item If the {\tt one\_row\_only} flag is set to {\tt
UNPACK\_CUTS\_MULTIPLE}, then the user can generate as many
constraints (even zero!) from a cut as she wants (this way she can
lift the cuts, thus adjusting them for the current LP).
However, if the flag is set to {\tt UNPACK\_CUTS\_SINGLE}, then for
each cut the user must generate a unique row, the same one that had
been generated from the cut before. (The flag is set to this value
only when regenerating a search tree node.)

\item The {\tt from} argument can take on six different values:
{\tt CUT\_FROM\_CG}, {\tt CUT\_FROM\_CP}, {\tt CUT\_FROM\_TM}, {\tt
CUT\_LEFTOVER} (these are cuts from a previous LP relaxation that are
still in the local pool), {\tt CUT\_NOT\_IN\_MATRIX\_SLACK} and {\tt
CUT\_VIOLATED\_SLACK} indicating where the cut came from. This might
be useful in deciding whether to lift the cut or not.

\item The {\tt matind} fields of the rows must be filled with indices with
respect to the position of the variables in {\tt **vars}.

\item {\bf Warning:} For each row, the user must make sure that the
cut the row was generated from (and can be uniquely regenerated from
if needed later) is safely stored in the {\tt 
\htmlref{waiting\_row}{waiting_row}} structure. \BB\ will free the entries in
{\tt cuts} after this function returns. If a row is generated from a
cut in {\tt cuts} (and not from a lifted cut), the user has the option
of physically copying the cut into the corresponding part of the
{\tt \htmlref{waiting\_row}{waiting_row}} structure, or copying the
pointer to the cut into the {\tt \htmlref{waiting\_row}{waiting_row}}
structure and erasing the pointer in {\tt cuts}. If a row is generated
from a lifted cut, the user should store a copy of the lifted cut in
the corresponding part of {\tt \htmlref{waiting\_row}{waiting_row}}.

\end{itemize}

\ed

\vspace{1ex}


%%%%%%%%%%%%%%%%%%%%%%%%%%%%%%%%%%%%%%%%%%%%%%%%%%%%%%%%%%%%%%%%%%%%%%%%%%%%%
% user_send_lp_solution
%%%%%%%%%%%%%%%%%%%%%%%%%%%%%%%%%%%%%%%%%%%%%%%%%%%%%%%%%%%%%%%%%%%%%%%%%%%%%

\functiondef{user\_send\_lp\_solution} 
\label{user_send_lp_solution}
\sindex[cf]{user\_send\_lp\_solution} 
\begin{verbatim}
int user_send_lp_solution(void *user, int varnum, var_desc **vars, 
                          double *x, int where)
\end{verbatim}

\bd
\describe

This function is only used in the case of parallel execution. 
The user has the option to send the LP solution to either the cut pool
or the cut generator in some user-defined form if desired. There are
two default options---sending the indices and values for all nonzero
variables ({\tt SEND\_NONZEROS}) and sending the indices and values
for all fractional variables ({\tt SEND\_FRACTIONS}).

\args

\bt{llp{285pt}}
{\tt void *user} &  IN & Pointer to the user-defined LP data structure. \\
& & \\
{\tt int varnum} & IN & The number of variables currently in the LP
relaxation. (The length of the {\tt *vars} and {\tt x} arrays.) \\
{\tt var\_desc **vars} & IN & The variables currently in the LP relaxation.\\
{\tt double *x} & IN & Values of the above variables.\\
{\tt int where} & IN & Where the solution is to be sent---{\tt
LP\_SOL\_TO\_CG} or {\tt LP\_SOL\_TO\_CP}. \\
\et

\returns

\bt{lp{315pt}}
{\tt USER\_ERROR} & Error. No message will be sent. \\
{\tt USER\_SUCCESS} & User packed and sent the message. \\
{\tt USER\_DEFAULT} & Regulated by the {\tt pack\_lp\_solution\_default}
parameter, initially set to {\tt SEND\_NOZEROS}. \\
{\tt SEND\_NONZEROS} & Send user indices and values of variables
at nonzero level.\\
{\tt SEND\_FRACTIONS} & Send user indices and values of variables
at fractional level.\\
\et

\item[Wrapper invoked from:] {\tt fathom\_branch()} after an LP relaxation has
been solved. The message is always sent to the cut generator (if there is
one). The message is sent to the cut pool if a search tree node at the top of
a chain is being processed (except at the root in the first phase), or if a
given number ({\tt cut\_pool\_check\_freq}) of LP relaxations have
been solved since the last check.

\item[Note:] \hfill

The wrapper automatically packs the level, index, and iteration number
corresponding to the current LP solution within the current search
tree node, as well as the objective value and upper bound in case the
solution is sent to a cut generator. This data will be unpacked by
\BB\ on the receiving end, the user will have to unpack there exactly
what he has packed here.

\ed
\vspace{1ex}


%%%%%%%%%%%%%%%%%%%%%%%%%%%%%%%%%%%%%%%%%%%%%%%%%%%%%%%%%%%%%%%%%%%%%%%%%%%%%
% user_logical_fixing
%%%%%%%%%%%%%%%%%%%%%%%%%%%%%%%%%%%%%%%%%%%%%%%%%%%%%%%%%%%%%%%%%%%%%%%%%%%%%

\functiondef{user\_logical\_fixing}
\sindex[cf]{user\_logical\_fixing}
\begin{verbatim}
int user_logical_fixing(void *user, int varnum, var_desc **vars, 
                        double *x, char *status, int *num_fixed)
\end{verbatim}

\bd
\describe

Logical fixing is modifying the stati of variables based on logical
implications derived from problem-specific information. In this
function the user can modify the status of any variable. Valid stati
are: {\tt NOT\_FIXED}, {\tt TEMP\_FIXED\_TO\_LB}, {\tt
PERM\_FIXED\_TO\_LB}, {\tt TEMP\_FIXED\_TO\_UB} and {\tt
PERM\_FIXED\_TO\_UB}. Be forewarned that fallaciously fixing a
variable in this function can cause the algorithm to terminate
improperly. Generally, a variable can only be fixed permanently if the
matrix is {\em full} at the time of the fixing (i.e. all variables
that are not fixed are in the matrix). There are no default options.

\args

\bt{llp{260pt}}
{\tt void *user} &  IN & Pointer to the user-defined LP data structure. \\
& & \\
{\tt int varnum} & IN & The number of variables currently in the LP
relaxation. (The length of the {\tt *vars} and {\tt x} arrays.) \\
{\tt var\_desc **vars} & IN & The variables currently in the LP relaxation.\\
{\tt double *x} & IN & Values of the above variables.\\
{\tt char *status} & INOUT & Stati of variables currently in the LP
relaxation. \\
{\tt int *num\_fixed} & OUT & Number of fixed variables.\\
\et

\returns

\bt{lp{300pt}}
{\tt USER\_ERROR} & Error. Ignored by \BB. \\
{\tt USER\_SUCCESS} & User changed the stati of the variables she
wanted. \\
{\tt USER\_DEFAULT} & No logical fixing rules are implemented. \\
\et

\item[Wrapper invoked from:] {\tt fix\_variables()} after doing reduced cost
fixing, but only when a specified number of variables have been fixed
by reduced cost (see LP parameter settings).

\ed
\vspace{1ex}

%%%%%%%%%%%%%%%%%%%%%%%%%%%%%%%%%%%%%%%%%%%%%%%%%%%%%%%%%%%%%%%%%%%%%%%%%%%%%
% user_generate_column
%%%%%%%%%%%%%%%%%%%%%%%%%%%%%%%%%%%%%%%%%%%%%%%%%%%%%%%%%%%%%%%%%%%%%%%%%%%%%

\functiondef{user\_generate\_column}
\sindex[cf]{user\_generate\_column}
\begin{verbatim}
int user_generate_column(void *user, int generate_what, int cutnum, 
                         cut_data **cuts, int prevind, int nextind, 
                         int *real_nextind, double *colval, 
                         int *colind, int *collen, double *obj,
                         double *lb, double *ub)
\end{verbatim}

\bd
\describe

This function is called when pricing out the columns that are not
already fixed and are not explicitly represented in the matrix. Only
the user knows the explicit description of these columns. When a
missing variable need to be priced, the user is asked to provide the
corresponding column. \BB\ scans through the known variables in the
order of their user indices. After testing a variable in the matrix
({\tt prevind}), \BB\ asks the user if there are any missing variables
to be priced before the next variable in the matrix ({\tt nextind}).
If there are missing variables before {\tt nextind}, the user has to
supply the user index of the real next variable ({\tt real\_nextind})
along with the corresponding column. Occasionally \BB\ asks the user to
simply supply the column corresponding to {\tt nextind}. The {\tt
generate\_what} flag is used for making a distinction between the two
cases: in the former case it is set to {\tt GENERATE\_REAL\_NEXTIND}
and in the latter it is set to {\tt GENERATE\_NEXTIND}.

\newpage
\args

\bt{llp{260pt}}
{\tt void *user} &  IN & Pointer to the user-defined LP data structure. \\
& & \\
{\tt int generate\_what} & IN & {\tt GENERATE\_NEXTIND} or {\tt
GENERATE\_REAL\_NEXTIND} (see description above). \\
{\tt int cutnum} & IN & The number of added rows in the LP formulation
(i.e., the total number of rows less the number of base constraints).
This is the length of the {\tt **cuts} array.\\ 
{\tt cut\_data **cuts} & IN & Description of the cuts corresponding to
the added rows of the current LP formulation. The user is supposed to
know about the cuts corresponding to the base constraints. \\
{\tt int prevind} & IN & The last variable processed ($-1$ if there was
none) by \BB.\\
{\tt int nextind} & IN & The next variable ($-1$ if there are none)
known to \BB.\\
& & \\
{\tt int *real\_nextind} & OUT & Pointer to the user index of the next
variable ($-1$ if there is none). \\
{\tt double *colval} & OUT & Values of the nonzero entries in the
column of the next variable. (Sufficient space is already allocated for this
array.) \\
{\tt int *colind} & OUT & Row indices of the nonzero entries in the column.
(Sufficient space is already allocated for this array.) \\
{\tt int *collen} & OUT & The length of the {\tt colval} and {\tt
colind} arrays.\\
{\tt double *obj} & OUT & Objective coefficient corresponding to the
next variable.\\
{\tt double *lb} & OUT & Lower bound of the next variable.\\
{\tt double *ub} & OUT & Upper bound of the next variable.\\
\et

\returns

\bt{lp{300pt}}
{\tt USER\_ERROR} & Error. The LP process is aborted.\\
{\tt USER\_SUCCESS} & User filled out {\tt *real\_nextind} and
generated its column if needed.\\
{\tt USER\_DEFAULT} & No column generation is done. \\
\et

\item[Wrapper invoked from:] {\tt price\_all\_vars()} and {\tt
restore\_lp\_feasibility()}.

\item[Note:] \hfill

{\tt colval}, {\tt colind}, {\tt collen} and {\tt obj} do not need to
be filled out if {\tt real\_nextind} is the same as {\tt nextind} and {\tt
generate\_what} is {\tt GENERATE\_REAL\_NEXTIND}.

\ed
\vspace{1ex}

%%%%%%%%%%%%%%%%%%%%%%%%%%%%%%%%%%%%%%%%%%%%%%%%%%%%%%%%%%%%%%%%%%%%%%%%%%%%%
% user_void generate_cuts_in_lp
%%%%%%%%%%%%%%%%%%%%%%%%%%%%%%%%%%%%%%%%%%%%%%%%%%%%%%%%%%%%%%%%%%%%%%%%%%%%%

\functiondef{user\_generate\_cuts\_in\_lp}
\label{user_generate_cuts_in_lp}
\sindex[cf]{user\_generate\_cuts\_in\_lp}
\begin{verbatim}
int user_generate_cuts_in_lp(void *user, LPdata *lp_data, int varnum,
                             var_desc **vars, double *x, int *new_row_num,
                             cut_data ***cuts)
\end{verbatim}

\bd
\describe

The user might decide to generate cuts directly within the LP module instead
of using the cut generator. This can be accomplished either through a call to
this function or simply by configuring \BB\ such that the cut generator is
called directly from the LP solver. One example of when this might be done is
when generating Gomory cuts or something else that requires knowledge of the
current LP tableau. The user must return the list of generated cuts by
allocating an array of \texttt{cut\_data} structures and setting
\texttt{*cuts} to point to this array. Post-processing consists of
checking if any of the new cuts are already in the local pool (or dominated by
a cut in the local pool).

\args

\bt{llp{225pt}}
{\tt void *user} &  IN & Pointer to the user-defined LP data structure. \\
& & \\
{\tt LPdata *lp\_data} & IN & A pointer to SYMPHONY's internal data structure for
storing the LP relaxation and related data. \\
{\tt int varnum} & IN & The number of variables currently in the LP
relaxation. (The length of the {\tt *vars} and {\tt x} arrays.) \\
{\tt var\_desc **vars} & IN & The variables currently in the LP relaxation.\\
{\tt double *x} & IN & Values of the above variables.\\
{\tt int *new\_row\_num} & OUT & The number of cuts generated. \\
{\tt cut\_data ***cuts} & OUT & The cuts and the corresponding rows. \\
\et

\returns

\bt{lp{330pt}}
{\tt USER\_ERROR} & Error. Interpreted as if no cuts were generated. \\
{\tt USER\_SUCCESS} & Cuts were generated. \\
{\tt USER\_DEFAULT} & No cuts were generated. By default, SYMPHONY uses the
CGL to generate generic cuts, according to parameter settings. \\
{\tt GENERATE\_CGL\_CUTS} & Generate generic CGL cuts, according to parameter
settings. \\
{\tt DO\_NOT\_GENERATE\_CGL\_CUTS} & No additional cuts are generated. \\
\et

\postp

\BB\ checks if any of the newly generated cuts are already in the
local pool. 

\item[Wrapper invoked from:] {\tt receive\_cuts()} before the cuts from the CG
module are received. Since the user will probably use this function
to generate tableau-dependent cuts, it is highly unlikely that any of
the new cuts would already be in the pool. Therefore the user will
probably return {\tt USER\_AND\_PP} to force \BB\ to skip post-processing.

\item[Notes:] \hfill \\

\begin{itemize}
\vspace{-3ex}
\item Just like in {\tt \htmlref{user\_unpack\_cuts()}{user_unpack_cuts}}, 
the user has to allocate space for the rows.
\item Unless the {\tt name} field of a cut is explicitly set to 
{\tt CUT\_\_SEND\_TO\_CP}, \BB\ will assume that the cut is locally
valid only and set that field to {\tt CUT\_\_DO\_NOT\_SEND\_TO\_CP}.
\end{itemize}
\ed
\vspace{1ex}

%%%%%%%%%%%%%%%%%%%%%%%%%%%%%%%%%%%%%%%%%%%%%%%%%%%%%%%%%%%%%%%%%%%%%%%%%%%%%
% user_print_stat_on_cuts_added
%%%%%%%%%%%%%%%%%%%%%%%%%%%%%%%%%%%%%%%%%%%%%%%%%%%%%%%%%%%%%%%%%%%%%%%%%%%%%

\functiondef{user\_print\_stat\_on\_cuts\_added}
\sindex[cf]{user\_print\_stat\_on\_cuts\_added}
\begin{verbatim}
int user_print_stat_on_cuts_added(void *user, int rownum, waiting_row **rows)
\end{verbatim}

\bd
\describe

The user can print out some information (if he wishes to) on the cuts that
will be added to the LP formulation. The default is to print out the number of
cuts added. 

\args

\bt{llp{265pt}}
{\tt void *user} &  IN & Pointer to the user-defined LP data structure. \\
& & \\
{\tt int rownum} & IN & The number of cuts added. \\
{\tt waiting\_row **rows} & IN & Array of waiting rows containing the
cuts added. \\
\et

\returns

\bt{lp{300pt}}
{\tt USER\_ERROR} & Revert to default. \\
{\tt USER\_SUCCESS} & User printed whatever he wanted. \\
{\tt USER\_DEFAULT} & Print out the number of cuts added. \\
\et

\item[Wrapper invoked from:] {\tt add\_best\_waiting\_rows()} after it has been decided how many cuts to
add and after the cuts have been selected from the local pool.

\ed
\vspace{1ex}

%%%%%%%%%%%%%%%%%%%%%%%%%%%%%%%%%%%%%%%%%%%%%%%%%%%%%%%%%%%%%%%%%%%%%%%%%%%%%
% user_purge_waiting_rows
%%%%%%%%%%%%%%%%%%%%%%%%%%%%%%%%%%%%%%%%%%%%%%%%%%%%%%%%%%%%%%%%%%%%%%%%%%%%%

\functiondef{user\_purge\_waiting\_rows}
\sindex[cf]{user\_purge\_waiting\_rows}
\begin{verbatim}
int user_purge_waiting_rows(void *user, int rownum,
                            waiting_row **rows, char *delete_rows)
\end{verbatim}

\bd
\describe

The local pool is purged from time to time to control its size. In
this function the user has the power to decide which cuts to purge
from this pool if desired. To mark the {\tt i$^{\tt th}$} waiting row
(an element of the pre-pool) for removal she has to set {\tt
delete\_rows[i]} to be {\tt TRUE} ({\tt delete\_rows} is allocated before the
function is called and its elements are set to {\tt FALSE} by
default).\\
\\
Post-processing consists of actually deleting those entries from the waiting
row list and compressing the list. The default is to discard the least
violated waiting rows and keep no more than what can be added in the next
iteration (this is determined by the {\tt max\_cut\_num\_per\_iter}
parameter).

\args

\bt{llp{255pt}}
{\tt void *user} &  IN & Pointer to the user-defined LP data structure. \\
& & \\
{\tt int rownum} & IN & The number of waiting rows. \\
{\tt waiting\_row **rows} & IN & The array of waiting rows.\\
{\tt char *delete\_rows} & OUT & An array of indicators showing which waiting 
rows are to be deleted. \\ 
\et

\returns

\bt{lp{330pt}}
{\tt USER\_ERROR} & Purge every single waiting row. \\
{\tt USER\_SUCCESS} & The user marked in {\tt delete} the rows to be deleted. 
\\
{\tt USER\_DEFAULT} & Described above. \\
\et

\postp

The marked rows are deleted.

\item[Wrapper invoked from:] {\tt receive\_cuts()} after cuts have been added.

\ed

\vspace{1ex}

%%%%%%%%%%%%%%%%%%%%%%%%%%%%%%%%%%%%%%%%%%%%%%%%%%%%%%%%%%%%%%%%%%%%%%%%%%%%%
% user_free_lp
%%%%%%%%%%%%%%%%%%%%%%%%%%%%%%%%%%%%%%%%%%%%%%%%%%%%%%%%%%%%%%%%%%%%%%%%%%%%%

\functiondef{user\_free\_lp}
\sindex[cf]{user\_free\_lp}
\begin{verbatim}
int user_free_lp(void **user)
\end{verbatim}

\bd

\describe

The user has to free all the data structures within {\tt *user}, and also free
{\tt user} itself. The user can use the built-in macro {\tt FREE} that checks
the existence of a pointer before freeing it. 

\args

\bt{llp{250pt}}
{\tt void **user} & INOUT & Pointer to the user-defined LP data structure. \\
\et

\returns

\bt{lp{300pt}}
{\tt USER\_ERROR} & Error. \BB\ ignores error message. \\
{\tt USER\_SUCCESS} & User freed everything in the user space. \\
{\tt USER\_DEFAULT} & There is no user memory to free. \\
\et

\item[Wrapper invoked from:] {\tt lp\_close()} at module shutdown.

\ed

\vspace{1ex}

%begin{latexonly}
\ed
%end{latexonly}


