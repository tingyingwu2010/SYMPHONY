%===========================================================================%
%                                                                           %
% This file is part of the documentation for the SYMPHONY MILP Solver.      %
%                                                                           %
% SYMPHONY was jointly developed by Ted Ralphs (tkralphs@lehigh.edu) and    %
% Laci Ladanyi (ladanyi@us.ibm.com).                                        %
%                                                                           %
% (c) Copyright 2005-2006 Mike Trick and Ted Ralphs. All Rights Reserved.   %
%                                                                           %
% SYMPHONY is licensed under the Common Public License. Please see          %
% accompanying file for terms.                                              %
%                                                                           %
%===========================================================================%

\documentclass[11pt]{article}

\setlength{\evensidemargin}{.25in}
\setlength{\oddsidemargin}{.25in}
\setlength{\textwidth}{6.0in}
\setlength{\parindent}{0in}
\setlength{\parskip}{0.1in}
\setlength{\topmargin}{0in}
\setlength{\textheight}{8.5in}

\usepackage{epsfig}
\usepackage{amssymb}

\newtheorem{theorem}{Theorem}
\newtheorem{acknowledgement}[theorem]{Acknowledgment}
\newtheorem{algorithm}[theorem]{Algorithm}
\newtheorem{axiom}[theorem]{Axiom}
\newtheorem{case}[theorem]{Case}
\newtheorem{claim}[theorem]{Claim}
\newtheorem{conclusion}[theorem]{Conclusion}
\newtheorem{condition}[theorem]{Condition}
\newtheorem{conjecture}[theorem]{Conjecture}
\newtheorem{corollary}[theorem]{Corollary}
\newtheorem{criterion}[theorem]{Criterion}
\newtheorem{definition}{Definition}
\newtheorem{example}[theorem]{Example}
\newtheorem{exercise}[theorem]{Exercise}
\newtheorem{lemma}[theorem]{Lemma}
\newtheorem{notation}[theorem]{Notation}
\newtheorem{problem}[theorem]{Problem}
\newtheorem{proposition}[theorem]{Proposition}
\newtheorem{remark}[theorem]{Remark}
\newtheorem{solution}[theorem]{Solution}
\newtheorem{summary}[theorem]{Summary}
\newenvironment{proof}[1][Proof]{\textbf{#1.} }{\ \rule{0.5em}{0.5em}}
\renewcommand{\Re}{\mathbb{R}}

%FIXME: Comment on number of rows and number of columns mixed up.
%FIXME: Do we even need the number of columns? Can be calcuated.
%FIXME; Shouldn't number of rows be 2*nodes?
%FIXME: Formula for computing index of a matching?
%FIXME: Change node1 to assign1.

\begin{document}

\title{Simple Walkthrough of Building a Solver with SYMPHONY \thanks{The
original version of this document was submitted by Michael Trick, but it has
since been updated by Menal Guzelsoy to correspond to version 5.0 of
SYMPHONY.}}

\author{Michael Trick\thanks{Graduate School of Industrial Administration,
Carnegie Mellon University, Pittsburgh, PA 15213 \texttt{trick@cmu.edu},
\texttt{http://mat.gsia.cmu.edu/trick}} and Menal Guzelsoy\thanks{Department
of Industrial and Systems Engineering, Lehigh University, Bethlehem, PA 18017,
{\tt megb@lehigh.edu}}}

\date{November 3, 2004}
\maketitle
\thispagestyle{empty}

SYMPHONY is a callable library that includes a set of user callback routines
to allow it to solve generic MIPs, as well as easily create custom
branch-cut-price solvers. Having been fully integrated with COIN, SYMPHONY is
capable to use CPLEX, OSL, CLP, GLPK, DYLP, SOPLEX, VOL and XPRESS-MP through
the COIN/OSI interface (first two can also be used through the built-in APIs
without using COIN/OSI). The SYMPHONY system includes sample solvers for
numerous applications: Vehicle Routing Problem (VRP), Capacitated Node Routing
Problem (CNRP), Multi-Criteria Knapsack Problem (MCKP), Mixed Postman Problem
(MPP), Set Partitioning Problem (basic and advanced solvers).  These
applications are extremely well done, but can be difficult to
understand because of their complexity.

Here is a walkthrough for a very simple application built using
SYMPHONY. Rather than presenting the code in its final version, I will go
through the steps that I went through. Note that some of the code is lifted
from the vehicle routing application. This code is designed to be a sequential
code. The MATCH application itself is available for download at
\texttt{http://www.branchandcut.org/MATCH}.

Our goal is to create a minimum matching on a complete graph.  Initially we
will just formulate this as an integer program with one variable for each
possible pair that can be matched. Then we will include a set of constraints
that can be added by cut generation.

I begin with the template code in the \texttt{USER} subdirectory included with
SYMPHONY. This gives stubs for each user callback routine. First, I need to
define a data structure for describing an instance of the matching problem. We
use the template structure \texttt{USER\_PROBLEM} in the file
\texttt{include/user.h} for this purpose.  To describe an instance, we just
need the number of nodes and the cost matrix. In addition, we also need a way
of assigning an index to each possible assignment. Here is the data
structure: 
\newpage
\begin{verbatim}
typedef struct USER_PROBLEM{
   int		    numnodes;
   int		    cost[MAXNODES][MAXNODES];
   int		    match1[MAXNODES*(MAXNODES-1)/2];
   int		    match2[MAXNODES*(MAXNODES-1)/2]; 
   int              index[MAXNODES][MAXNODES];
}user_problem;
\end{verbatim}

The fields \texttt{match1}, \texttt{match2}, and
\texttt{index} will be used later in the code in order to map variables to the
corresponding assignment and vice versa. 

Next, we need to read in the problem instance. We could implement this
function within the \texttt{user\_io()} callback function (see the file
\texttt{user\_master.c}). However, in order to show how it can be done
explicitly, we will define our own function \texttt{match\_read\_data()} in
\texttt{user\_main.c} to fill in the user data structure and then use
\texttt{sym\_set\_user\_data()} to pass this structure to SYMPHONY. The
template code already provides basic command-line options for the user. The
``-F'' flag is used to specify the location of a data file, from which we will
read in the data. The datafile contains first the number of nodes in the graph
(\texttt{nnodes}) followed by the pairwise cost matrix (nnode by nnode).  We
read the file in with the \texttt{match\_read\_data()} routine in
\texttt{user\_main.c}:

%FIXME: Here, we don't need to pass in the user data structure...just return
%it. We also don't need to pass in the sym_enviroment 

\begin{verbatim}
int match_read_data(user_problem *prob, char *infile)
{
   int i, j;
   FILE *f = NULL;

   if ((f = fopen(infile, "r")) == NULL){
      printf("main(): user file %s can't be opened\n", infile);
      return(ERROR__USER); 
   }

   /* Read in the costs */
   fscanf(f,"%d",&(prob->numnodes));
   for (i = 0; i < prob->numnodes; i++)
      for (j = 0; j < prob->numnodes; j++)
	 fscanf(f, "%d", &(prob->cost[i][j]));
   
   return (FUNCTION_TERMINATED_NORMALLY);
}
\end{verbatim}   

We can now construct the integer program itself. This is done by specifying
the constraint matrix and the rim vectors in sparse format. We will have a
variable for each possible assignment $(i,j)$ with $i<j$. We have a constraint
for each node $i$, so it can only me matched to one other node.

We define the IP in our other helper function \texttt{match\_load\_problem()}
in \texttt{user\_main.c}. In the first part of this routine, we will build a
description of the IP, and then in the second part, we will load this
representation to SYMPHONY through
\texttt{sym\_explicit\_load\_problem()}. Note that we could instead create a
description of each subproblem dynamically using the
\texttt{user\_create\_subproblem()} callback (see \texttt{user\_lp.c}), but
this is more complicated and unnecessary here.

\begin{verbatim}
int match_load_problem(sym_environment *env, user_problem *prob){
   
   int i, j, index, n, m, nz, *matbeg, *matind;
   double *matval, *lb, *ub, *obj, *rhs, *rngval;
   char *sense, *is_int;
   
   /* set up the inital LP data */
   n = prob->numnodes*(prob->numnodes-1)/2;
   m = 2 * prob->numnodes;
   nz = 2 * n;

   /* Allocate the arrays */
   matbeg  = (int *) malloc((n + 1) * ISIZE);
   matind  = (int *) malloc((nz) * ISIZE);
   matval  = (double *) malloc((nz) * DSIZE);
   obj     = (double *) malloc(n * DSIZE);
   lb      = (double *) calloc(n, DSIZE);
   ub      = (double *) malloc(n * DSIZE);
   rhs     = (double *) malloc(m * DSIZE);
   sense   = (char *) malloc(m * CSIZE);
   rngval  = (double *) calloc(m, DSIZE);
   is_int  = (char *) malloc(n * CSIZE);
   
   /* Fill out the appropriate data structures -- each column has
      exactly two entries */
   index = 0;
   for (i = 0; i < prob->numnodes; i++) {
      for (j = i+1; j < prob->numnodes; j++) {
	 prob->match1[index] = i; /*The first component of assignment 'index'*/
	 prob->match2[index] = j; /*The second component of assignment 'index'*/
	 /* So we can recover the index later */
	 prob->index[i][j] = prob->index[j][i] = index;
	 obj[index] = prob->cost[i][j]; /* Cost of assignment (i, j) */
	 is_int[index] = TRUE;
	 matbeg[index] = 2*index;
	 matval[2*index] = 1;
	 matval[2*index+1] = 1;
	 matind[2*index] = i;
	 matind[2*index+1] = j;
	 ub[index] = 1.0;
	 index++;
      }
   }
   matbeg[n] = 2 * n;
   
   /* set the initial right hand side */
   for (i = 0; i < m; i++) {
      rhs[i] = 1;
      sense[i] = 'E';
   }
   
   /* Load the problem to SYMPHONY */   
   sym_explicit_load_problem(env, n, m, matbeg, matind, matval, lb, ub, 
			     is_int, obj, 0, sense, rhs, rngval, true);
			     
   return (FUNCTION_TERMINATED_NORMALLY);

}
\end{verbatim}

Now, we are ready to gather everything in the \texttt{main()} routine in 
\texttt{user\_main()}. This will involve to create a SYMPHONY environment and 
a user data structure, read in the data, create the corresponding IP, 
load it to the environment and ask SYMPHONY to solve it 
(\texttt{CALL\_FUNCTION} is just a macro to take care of the return values):  

\begin{verbatim}
int main(int argc, char **argv)
{
   int termcode;
   char * infile;

   /* Create a SYMPHONY environment */
   sym_environment *env = sym_open_environment();

   /* Create the data structure for storing the problem instance.*/
   user_problem *prob = (user_problem *)calloc(1, sizeof(user_problem));
   
   CALL_FUNCTION( sym_set_user_data(env, (void *)prob) );
   CALL_FUNCTION( sym_parse_command_line(env, argc, argv) );
   CALL_FUNCTION( sym_get_str_param(env, "infile_name", &infile));
   CALL_FUNCTION( match_read_data(prob, infile) );
   CALL_FUNCTION( match_load_problem(env, prob) );
   CALL_FUNCTION( sym_solve(env) );
   CALL_FUNCTION( sym_close_environment(env) );
   return(0);
}
\end{verbatim}

OK, that's it. That defines an integer program, and if you compile and
optimize it, the rest of the system will come together to solve this problem.
Here is a data file to use:
\begin{verbatim}
6
0 1 1 3 3 3
1 0 1 3 3 3
1 1 0 3 3 3
3 3 3 0 1 1
3 3 3 1 0 1
3 3 3 1 1 0
\end{verbatim}

The optimal value is 5. To display the solution, we need to be able to map
back from variables to the nodes. That was the use of the \texttt{node1} and
\texttt{node2} parts of the \texttt{USER\_PROBLEM}. We can now use
\texttt{user\_display\_solution()} in \texttt{user\_master.c} to print 
out the solution:

\begin{verbatim}
int user_display_solution(void *user, double lpetol, int varnum, int *indices,
                          double *values, double objval)
{
   /* This gives you access to the user data structure. */
   user_problem *prob = (user_problem *) user;
   int index;
 
   for (index = 0; index < varnum; index++){
      if (values[index] > lpetol) {
          printf("%2d matched with %2d at cost %6d\n",
                prob->node1[indices[index]],
                prob->node2[indices[index]],
                prob->cost[prob->node1[indices[index]]]
                [prob->node2[indices[index]]]);
      }	   
   }
   
   return(USER_SUCCESS);
}
\end{verbatim}

We will now update the code to include a crude cut generator. Of course, I am
eventually aiming for a Gomory-Hu type odd-set separation (ala Gr\"otschel and
Padberg) but for the moment, let's just check for sets of size three with more
than value 1 among them (such a set defines a cut that requires at least one
edge out of any odd set). We can do this by brute force checking of triples,
as follows: 

\begin{verbatim}
int user_find_cuts(void *user, int varnum, int iter_num, int level,
                   int index, double objval, int *indices, double *values,
                   double ub, double etol, int *num_cuts, int *alloc_cuts, 
                   cut_data ***cuts)
{
   user_problem *prob = (user_problem *) user;
   double edge_val[200][200]; /* Matrix of edge values */
   int i, j, k, cutind[3];
   double cutval[3];
   
   int cutnum = 0;

   /* Allocate the edge_val matrix to zero (we could also just calloc it) */
   memset((char *)edge_val, 0, 200*200*ISIZE);
   
   for (i = 0; i < varnum; i++) {
      edge_val[prob->node1[indices[i]]][prob->node2[indices[i]]] = values[i];
   }
   
   for (i = 0; i < prob->nnodes; i++){
      for (j = i+1; j < prob->nnodes; j++){
	 for (k = j+1; k < prob->nnodes; k++) {
	    if (edge_val[i][j]+edge_val[j][k]+edge_val[i][k] > 1.0 + etol) {
	       /* Found violated triangle cut */
	       /* Form the cut as a sparse vector */
	       cutind[0] = prob->index[i][j];
	       cutind[1] = prob->index[j][k];
	       cutind[2] = prob->index[i][k];
	       cutval[0] = cutval[1] = cutval[2] = 1.0;
	       cg_add_explicit_cut(3, cutind, cutval, 1.0, 0, 'L',
				   TRUE, num_cuts, alloc_cuts, cuts);
	       cutnum++;
	       
	    }
	 }
      }
   }

   return(USER_SUCCESS);
}

\end{verbatim}

Note the call of \texttt{cg\_add\_explicit\_cut()}, which tells SYMPHONY about
any cuts found. If we now solve the matching problem on the sample data set,
the number of nodes in the branch and bound tree should just be 1 (rather than
3 without cut generation).

\end{document}
 
