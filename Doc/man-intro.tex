\section{Introducing SYMPHONY 5.0}
\label{whats-new}

Welcome to the SYMPHONY user's manual. Whether you are a new user or simply
upgrading to version 5.0, this manual will help you get started with what we
hope you will find to be a very useful framework for solving mixed-integr
linear programs either using the generic tools provided or by developing a
custom branch, cut, and price algorithm. There have been some very significant
developments since the last version of SYMPHONY was released. IN particular,
SYMPHONY is now a callable library with an interface whose look and feel is
similar to other popular solvers. This change allows SYMPHONY to be used in a
variety of new and powerful ways that were not possible before. For existing
users, there have been a few minor changes to the API needed to make SYMPHONY
thread-safe. Code written for previous versions of SYMPHONY will have to be
ported. Instructions for porting from previous version are contained in the
file \texttt{SYMPHONY-5.0/README-5.0}. As always, these changes have
undoubtedly introduced bugs. There are now an even larger number of
configurations in which SYMPHONY can be used and we have tested many of them,
but it is simply not possible to test them all. Please keep this in mind and
report all bugs that you find. Among the new enhancements and features are:

\begin{itemize}

\item SYMPHONY is now a C callable library with an interface whose look and
feel is similar to other popular solvers. This interface works for SYMPHONY's
built-in generic MILP solver, as well as any customized algorithm developed by
implementing one or more of SYMPHONY's user callback functions. The interface
is exactly the same for both sequential and parallel versions of the code.

\item The callable library also has a C++ interface conforming to COIN-OR's
Open Solver Interface standard for accessing LP and MILP solvers.

\item SYMPHONY has been made thread-safe in order to allow multiple
environments to be opened within a single executable.

\item It is now possible to stop SYMPHONY during the solution process and then
restart the computation later, even after modifying the problem data. The user
can also save warm start inforation outside the solver environment and then
reload it later into a different environment, in much the same way as can be
done with a simplex-based linear programming solver. This allows the user to
efficiently implement procedures, such as those for multi-criteria
optimization, in which a series of similar MILPs must be solved.

\item Along with the ability to perform warm starts, the user call also define
permanent cut pools that persist between solver calls. This is useful for
situations in which a series of MILPs needs to be solved and the cuts
generated during one solution call are still valid during later calls.

\item SYMPHONY now has the ability to enumerate the efficient solutions of a
bicriteria MILP if the user specifies a second objective function. This is
done using a new algorithm described in \cite{WCN} and takes advantage of the
warm starting capabilities of SYMPHONY.

\item SYMPHONY has a very rudimentary to perform sensitivity analysis for
MILP. This capability is till very much in the development stages, but is
present in version 5.0.

\end{itemize}

\section{How to Use This Manual}

The manual is divided into seven chapters. The first is the introduction,
which you are reading now. Chapter \ref{SYMPHONY-background} contains
background information. Those not familiar with the basic methodology of
branch, cut, and price should read these sections, especially Section
\ref{B&C-intro}, where we briefly describe the techniques involved. Chapter
\ref{API-overview} contains an overview of the API, both for the callable
library and for the user callback functions. Chapter \ref{SYMPHONY-design}
contains further depth and a more complete description of the design and
implementation of SYMPHONY. In Section \ref{design}, we describe the overall
design of without reference to the implementational details and with only
passing reference to parallelism. In Section \ref{modules}, we discuss the
details of the implementation. In Section \ref{parallelizing}, we briefly
discuss issues involved in parallel execution of SYMPHONY. It is not necessary
to read Chapters \ref{SYMPHONY-background} and \ref{SYMPHONY-design} before
undertaking development of a SYMPHONY application. Chapter
\ref{getting_started} describes how to install and compile SYMPHONY. Many
users will want to go straight to this section of the manusal to get started
quickly. Chapter \ref{SYMPHONY-development} describes in detail how to develop
a custom application using SYMPHONY. For those who are familiar with branch
and cut and want to get started quickly, proceed directly to to Section
\ref{getting_started} for information on getting started. Chapter
\ref{SYMPHONY-reference} contains reference material. Section
\ref{callable-library} contains a description of the interface for the
callable library, both for C and C++ environments. Section \ref{API} contains
a description of the user callback functions. SYMPHONY's parameters are
described in Section \ref{params}. Please note that for reference use, the
HTML version of this manual may be more practical, as the embedded hyperlinks
make it easier to navigate.
