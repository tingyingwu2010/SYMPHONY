%===========================================================================%
%                                                                           %
% This file is part of the documentation for the SYMPHONY MILP Solver.      %
%                                                                           %
% SYMPHONY was jointly developed by Ted Ralphs (tkralphs@lehigh.edu) and    %
% Laci Ladanyi (ladanyi@us.ibm.com).                                        %
%                                                                           %
% (c) Copyright 2000-2006 Ted Ralphs. All Rights Reserved.                  %
%                                                                           %
% SYMPHONY is licensed under the Common Public License. Please see          %
% accompanying file for terms.                                              %
%                                                                           %
%===========================================================================%

\section{Introducing SYMPHONY 5.1}
\label{whats-new}

Welcome to the SYMPHONY Version 5.1 user's manual. Whether you are a new user
or simply upgrading to 5.1, this manual will help you get started with what we
hope you will find to be a useful and powerful framework for solving
mixed-integer linear programs (MILP) either using the generic tools provided
or by developing a custom brancha and cut algorithm. The subroutines in the
\BB\ library comprise a state-of-the-art MILP solver with a modular design
that is easy to customize for various problem settings. All internal library
subroutines are generic---their implementation does not depend on the the
problem setting. SYMPHONY therefore works out of the box as a generic MILP
solver, with the ability to read both MPS files and GMPL (a subset of AMPL)
files, as well as interface with more powerful modeling environments, such as
FlopC++. As a blackbox solver, SYMPHONY can be invoked either from the command
line, through an interactive shell, or by linking to the provided callable
library. To develop a customized SYMPHONY application, various callbacks can
be written and parameters set that modify the default behavior of the
algorithm. The API for these subroutines is described in
Chapter~\ref{API-overview} and files containing function stubs are provided.

The callable library has both C and C++ interfaces whose look and feel are
similar to other popular solvers (see Sections \ref{C_Interface} and
\ref{C++_Interface} for the library routines). The interface is identical for
SYMPHONY's built-in generic MILP solver (both parallel and sequential
versions), as well as all custom applications developed by implementing one or
more of SYMPHONY's user callback functions. SYMPHONY can be built in a wide
variety of configurations, ranging from fully parallel to completely
sequential, depending on the user's needs. The library runs serially on almost
any platform, and can also run in parallel in either a fully distributed
environment (network of workstations) or a shared-memory environment simply by
changing a few configuration options (see Chapter~\ref{getting_started}). To
run in a distributed environment, the user must have installed the {\em
\htmladdnormallink{Parallel Virtual Machine}{http://www.ccs.ornl.gov/pvm/}}
(PVM), available for free from Oak Ridge National Laboratories. To run in a
shared-memory environment, the user must have installed an OpenMP compliant
compiler (gcc 4.2 is currently the only compiler tested and fully supported).

\section{What's New}

There have been some significant developments since the last version of
SYMPHONY was released. We have continued to develop some of the more unique
capabilities SYMPHONY provides, such as the ability to warm start MILP
computations, the ability to perform simple sensitivity analyses, and the
ability to solve bicriteria MILPs. Specifically, the new enhancements and
features include:

\begin{itemize}

\item As mentioned above, SYMPHONY now has an interactive optimizer that can
be used through a command shell. In both the sequential and parallel
configurations, the user can set parameters, load and solve instances
interactively, and display results and statistics.

\item SYMPHONY now supports automatic configuration using the new COIN-OR build
system and the GNU autotools.Using autotools utilities, it is now possible to
build SYMPHONY in most operating systems and with most common compilers
compilers without user intervention.

\item Both the distributed and shared memory parallel configurations are now
fully debugged, tested, and supported. The user can now build and execute
custom SYMPHONY applications in parallel, as well as solving generic MILPs in
parallel "out of the box."

\item There are now additional options for warm starting. The user can trim the
warm starting tree before starting to resolve a problem. More specifically,
the user can decide to initiate warm starting with a predefined partition of
the final branch-and-cut tree resulting from a previous solution procedure.
This partition can include either a number of nodes created first during the
solution procedure or all of the nodes above a given level of the tree.

\end{itemize}

Two features have also been deprecated and are no longer supported:

\begin{itemize}

\item The native interfaces to OSL and CPLEX are now deprecated and no longer
supported. These solvers can be called through the COIN-OR OSI interface.

\item Column generation functionality has also been officially deprecated. For
now, there are a number of other software packages that offer better
functionality than SYMPHONY for implementing branch and price algorithms.

\end{itemize}

There was one minor change to the user callback API from version 5.0 to 5.1.
The user can now execute a primal heuristic in the
\ptt{user\_is\_feasible()} callback and return the solution to SYMPHONY.
Several new subroutines were also added to the callable library API. See the
\texttt{README} file included with the distribution for more details.

\section{A Brief History}
\label{history}

Since the inception of optimization as a recognized field of study in
mathematics, researchers have been both intrigued and stymied by the
difficulty of solving many of the most interesting classes of discrete
optimization problems. Even combinatorial problems, though conceptually easy
to model as integer programs, have long remained challenging to solve in
practice. The last two decades have seen tremendous progress in our ability to
solve large-scale discrete optimization problems. These advances have
culminated in the approach that we now call {\it branch and cut}, a technique
(see \cite{Grotschel84cut,padb:branc,hoff:LP}) which brings the computational
tools of branch and bound algorithms together with the theoretical tools of
polyhedral combinatorics. Indeed, in 1998, Applegate, Bixby, Chv\'atal, and
Cook used this technique to solve a {\em Traveling Salesman Problem} instance
with 13,509 cities, a full order of magnitude larger than what had been
possible just a decade earlier \cite{concorde} and two orders of magnitude
larger than the largest problem that had been solved up until 1978. This feat
becomes even more impressive when one realizes that the number of variables in
the standard formulation for this problem is approximately the {\em square} of
the number of cities. Hence, we are talking about solving a problem with
roughly {\em 100 million variables}.

There are several reasons for this impressive progress. Perhaps the most
important is the dramatic increase in available computing power over the last
decade, both in terms of processor speed and memory. This increase in the
power of hardware has subsequently facilitated the development of increasingly
sophisticated software for optimization, built on a wealth of theoretical
results. As software development has become a central theme of optimization
research efforts, many theoretical results have been ``re-discovered'' in
light of their new-found computational importance. Finally, the use of
parallel computing has allowed researchers to further leverage their gains.

Because of the rapidly increasing sophistication of computational techniques,
one of the main difficulties faced by researchers who wish to apply these
techniques is the level of effort required to develop an efficient
implementation. The inherent need for incorporating problem-dependent methods
(most notably for dynamic generation of variables and cutting planes) has
typically required the time-consuming development of custom implementations.
Around 1993, this led to the development by two independent research groups of
software libraries aimed at providing a generic framework that users could
easily customize for use in a particular problem setting. One of these groups,
headed by J\"unger and Thienel, eventually produced ABACUS (A Branch And CUt
System) \cite{abacus1}, while the other, headed by the authors, produced what
was then known as COMPSys (Combinatorial Optimization Multi-processing
System). After several revisions to enable more broad functionality, COMPSys
became SYMPHONY (Single- or Multi-Process Optimization over Networks). A
version of SYMPHONY written in C++, which we call COIN/BCP has also been
produced at IBM under the COIN-OR project \cite{coin-or}. The COIN/BCP package
takes substantially the same approach and has the same functionality as
SYMPHONY, but has extended SYMPHONY's capabilities in some areas.

\section{Related Work}
\label{related}

The 1990's witnessed a broad development of software for discrete
optimization. Almost without exception, these new software packages were based
on the techniques of branch, cut, and price. The packages fell into two main
categories---those based on general-purpose algorithms for solving
mixed-integer linear programs (MILPs) (without the use of special structure)
and those facilitating the use of special structure by interfacing with
user-supplied, problem-specific subroutines. We will call packages in this
second category {\em frameworks}. There have also been numerous
special-purpose codes developed for use in particular problem settings.

Of the two categories, MILP solvers are the most common. Among the dozens of
offerings in this category are MINTO \cite{MINTO}, MIPO \cite{MIPO}, bc-opt
\cite{bc-opt}, and SIP \cite{SIP}. Generic frameworks, on the other hand, are
far less numerous. The three frameworks we have already mentioned (SYMPHONY,
ABACUS, and COIN/BCP) are the most full-featured packages available. Several
others, such as MINTO, originated as MILP solvers but have the capability of
utilizing problem-specific subroutines. CONCORDE \cite{concorde, concorde2}, a
package for solving the {\em Traveling Salesman Problem} (TSP), also deserves
mention as the most sophisticated special-purpose code developed to date.

Other related software includes several frameworks for implementing parallel
branch and bound. Frameworks for general parallel branch and bound include
PUBB \cite{PUBB}, BoB \cite{BoB}, PPBB-Lib \cite{PPBB-Lib}, and PICO
\cite{PICO}. PARINO \cite{PARINO} and FATCOP \cite{chen:fatcop2} are parallel
MILP solvers.

\section{How to Use This Manual}

The manual is divided into six chapters. The first is the introduction, which
you are reading now. Chapter \ref{getting_started} describes how to install
SYMPHONY from either a source or binary distribution. If you have already
managed to get SYMPHONY running using the instructions in the \texttt{README}
file, you might want to skip to the next chapter. However, keep in mind that
the manual contains additional details for customizing your build.
\ref{API-overview} contains an overview of how to use in all three major
modes---as a black-box solver through the interactive shell or on the command
line, as a callable library, and as a customizable framework. Chapter
\ref{SYMPHONY-design} contains further depth and a more complete technical
description of the design and implementation of SYMPHONY. In Section
\ref{design}, we describe the overall design of SYMPHONY without reference to
the implementational details and with only passing reference to parallelism.
In Section \ref{modules}, we discuss the details of the implementation. In
Section \ref{parallelizing}, we briefly discuss issues involved in parallel
execution of SYMPHONY. Chapter \ref{SYMPHONY-development} describes in detail
how to develop a custom application using SYMPHONY. Note that it is not
necessary to read Chapter \ref{SYMPHONY-design} before undertaking development
of a SYMPHONY application, but it may help. Chapter \ref{SYMPHONY-reference}
contains reference material. Section \ref{C_Interface} contains a description
of the native C interface for the callable library. Section
\ref{C++_Interface} contains a description of the interface for C++
environments. Section \ref{API} contains a description of the user callback
functions. SYMPHONY's parameters are described in Section \ref{params}. For
reference use, the HTML version of this manual may be more practical, as the
embedded hyperlinks make it easier to navigate.
