\subsection{Orienting Yourself}

The easiest way to get oriented is to examine the organization of the source
files (note that file names will be given Unix-style). When you unpack the
\BB\ distribution, you will notice that the source files are organized along
the lines of the modules.  There is a separate directory for each
module---master ({\tt Master}), tree manager ({\tt TreeManager}), cut
generator ({\tt CutGen}), cut pool ({\tt CutPool}), and LP solver ({\tt
LP}). In addition, there is a directory called {\tt DrawGraph} and a directory
called {\tt Common} that also contain source files. The {\tt DrawGraph}
directory provides an interface from \BB\ to the {\em Interactive Graph
Drawing} software package developed by Marta Es\"o. This is an excellent
utility for graphical display and debugging. The {\tt Common} directory
contains source code for functions used by multiple modules.

Within each module's directory, there is a primary source file
containing the function {\tt main()} (named {\tt *.c} where * is the
module name), a source file containing functions related to
inter-process communication (named {\tt *\_proccomm.c}) and a file
containing general subroutines used by the module (named {\tt
*\_func.c}). The master is the exception and is organized slightly
differently. The LP process source code is further subdivided due to
the sheer number of functions.

The {\tt include} directory contains the header files. Corresponding
to each module, there are three header files, one containing internal
data structures and function prototypes associated with the module
(named {\tt *.h} where * is the module name), one containing the data
structures for storing the parameters (these are also used by the
master process), and the third containing the function prototypes for
the user functions (name {\tt *\_u.h}). By looking at the header files, you
should get a general idea of how things are laid out.

In addition to the subdirectories corresponding to each module, there is a
subdirectory called \texttt{SYMPHONY-4.0/USER}, which contains the files
needed for customizing the behaviors of the modules. Before beginning
customization, it is recommended to make a copy of the directory
\texttt{SYMPHONY-4.0/USER} that will be used as a template for creating your
customizesd solver. In this directory and its subdirectories, which mirror the
subdirectories of SYMPHONY itself, ach file contains function stubs that can
be filled in to create a new custom application. There is one file for each
module, initially called \texttt{SYMPHONY-4.0/USER/*/user\_*.c}, where
\texttt{*} is the name of the module. The primary thing that you, as the user,
need to understand to build a custom application is how to fill in these
stubs. That is what the second section of this manual is about.

\subsection{Writing the User Functions}

For each module, all user functions are invoked from so-called \emph{wrapper
functions} that provide the interface and also performs a default action if
the user chooses not to override it. Although SYMPHONY is written in C, the
wrapper functions provide a C++-style interface in whisch the user can either
accept the default action or override it. Each wrapper function is named {\tt
*\_u()} , where {\tt *} is the name of the corresponding user function, and is
defined in a file called {\tt *\_wrapper.c}. The wrapper function first
collects the necessary data and hands it to the user by calling the user
function. Based on the return value from the user, the wrapper then performs
any necessary post-processing. All user functions have default options, so
that SYMPHONY now acts as a generic MIP solver out of the box.

In the next section, the user functions are described in detail.
The name of every user written function starts with {\tt user\_}.
There are three kinds of arguments:
\begin{description}
\item[\rm IN:] An argument containing information that the user might need
to perform the function.
\item[\rm OUT:] A pointer to an argument in which the user should
return a result (requested data, decision, etc.) of the function. 
\item[\rm INOUT:] An argument which contains information the user might need,
but also for which the user can change the value.
\end{description}
The return values for most function are as follows:
\begin{description}
\item[Return values:] \hfill

\begin{tabular}{lp{310pt}} 

{\tt USER\_ERROR} & Error in the user function. Printing an error message is
the user's responsibility. Depending on the work the user function was
supposed to do, the error might be ignored (and some default option
used), or the process aborts. \\

{\tt USER\_SUCCESS} & The user function was implemented and executed correctly. \\

{\tt USER\_DEFAULT} & This option means that the user function was not
imlemented and that SYMPHONY should either execute a default subroutine (the
default is one of the built-in options, \BB\ decides which one to use based on
initial parameter settings and the execution of the algorithm) or else do
nothing, if execution of the subroutine is optional. \\

{\tt built\_in\_option1 } & \\
{\tt built\_in\_option2 } ... & The specified built-in option will be used.\\
\end{tabular}

\item[Notes:] \hfill
\begin{itemize}
\vspace{-3ex}

\item Sometimes an output is optional. This is always noted in the
function descriptions.

\item If an array has to be returned (i.e., the argument is {\tt type
**array}) then (unless otherwise noted) the user has to allocate space
for the array itself and set {\tt *array} to be the array allocated.
If an output array is optional then the user {\em must not} set {\tt
*array} for the array she is not going to fill up because this is how
\BB\ decides which optional arrays are filled up.

\item Some built-in options are implemented so that the user can invoke them
directly from the user function. This might be useful if, for example,
the user wants to use different built-in options at different stages
of the algorithm.
\end{itemize}

\end{description}

\subsection{Data Structures}

\subsubsection{Internal Data Structures}

With few exceptions, the data structures used internally by \BB\
are undocumented and most users will not need to access them
directly. However, if such access is desired, a pointer to the main data
structure used by each of the modules can be obtained simply by calling
the function {\tt get\_*\_ptr()} where * is the appropriate module (see the
header files). This function will return a pointer to the data
structure for the appropriate module. Casual users are advised against
modifying \BB's internal data structures directly.

\subsubsection{User-defined Data Structures}

The user can define her own data structure for each module to maintain problem
data and any other information the user needs access to in order to implement
functions to customize the solver. A pointer to this data structure is
maintained by \BB\ and is passed to the user as an argument to each user
function. Since \BB\ knows nothing about this data structure, it is up to the
user to allocate it and maintain it. The user must also implement a function
to free it. The functions for freeing the user data structures in each module
are called \texttt{user\_free\_*}, where \texttt{*} is the module. These
functions are called by SYMPHONY at the time when other data structures for
the modules are being freed and the module is being closed. By default, for
sequential computation, there is one common user data structure for all
modules and the pointer to that data structure is passed to all user
functions, regardless of the module. This setup should work fine for most
sequential applications. In parallel, however, poiners cannot be shared
between modules and data must be explicitly passed. IN this case, it is
sometimes more efficien to maintain in each module only the data necessary to
perform the functions of that module.

\subsection{Inter-process Communication for Distributed Computing}
\label{communication}
While the implementation of \BB\ strives to shield the user from having to
know anything about communications protocols or the specifics of inter-process
communication, it may be necessary for the user to pass information from one
module to another in order to implement a parallel application. For instance,
the user may want to pass data describing the problem instance to the LP
process after reading them in from a file in the master process. For the
purpose of passing user data from the master process to other processes, a
customization function called
\texttt{user\_send\_*\_data()} is provided in the master module, along with a
corresponding function called \texttt{user\_receive\_*\_data()} in the module
\texttt{*}. These two functions work in tandem to transport the user's data
from the maser, where it can be read in from a file, to the proper module for
processing. There are also a number of other tandem pairs of \emph{send} and
\emph{receive} functions that are used to transport user data from place to
place.

All data are sent in the form of arrays of either type {\tt char}, {\tt int},
or {\tt double}, or as strings. To send an array, the user has simply to
invoke the function {\tt send\_?\_array(? *array, int length)} where ?  is one
of the previously listed types. To receive that array, there is a
corresponding function called {\tt receive\_?\_array(?  *array, int
length)}. When receiving an array, the user must first allocate the
appropriate amount of memory. In cases where variable length arrays need to be
passed, the user must first pass the length of the array (as a separate array
of length one) and then the array itself. In the receive function, this allows
the length to be received first so that the proper amount of space can be
allocated before receiving the array itself. Note that data must be received
in exactly the same order as it was passed, as data is read linearly into and
out of the message buffer. The easiest way to ensure this is done properly is
to simply copy the send statements into the receive function and change the
function names. It may then be necessary to add some allocation statements in
between the receive function calls.

\subsection{The LP Engine}

\BB\ requires the use of a third-party callable library to solve the LP 
relaxations once they are formulated. Legacy interfaces to ILOG's
\htmladdnormallink{CPLEX}{http://www.cplex.com}$^{\copyright}$
and IBM's
\htmladdnormallink{OSL}{http://www-4.ibm.com/software/data/bi/osl/index.html}
are available. As of Version 4.0, the Open Solver Interface, avilable from
\htmladdnormallink{COIN}{http://www.coin-or.org}
\begin{latexonly} 
COIN (\texttt{http://www.coin-or.org})
\end{latexonly}.
Through the OSI, most commonly available LP solvers can be used. The list of
solvers with OSI interfaces currently numbers eight and includes both
commercial and open source alternatives. If the COIN libraries are used, make
sure to set the proper paths in the SYMPHONY make file.

\subsection{Cut Generation}

SYMPHONY now generates generic cutting planes using the Cut Generator Library,
also available from 
\htmladdnormallink{COIN}{http://www.coin-or.org}
\begin{latexonly} 
COIN (\texttt{http://www.coin-or.org})
\end{latexonly}.
The CGL can be used to generate cuts in cases where problem-specific cutting
planes are not available or not implemented yet. 

\subsection{Compiling Your Application}

\subsubsection{Unix Operating Systems}

Once the user functions are filled in, all that remains is to compile the
application. The distribution comes with two make files that facilitate this
process. The primary make file resides in the {\tt SYMPHONY-4.0/} directory.
The user make file resides in the user's subdirectory, initially called
\texttt{SYMPHONY-4.0/USER/}. THis subdirectory can be moved, as well as
renamed. There are a number of variables that must be set in the primary make
file. To modify the make files appropriately, see the instructions in Section
\ref{getting_started_unix}.

\paragraph{Working with PVM.}
\label{PVM}
To compile a distributed application, it is necessary to install PVM.
The current version of PVM can be obtained at {\tt
\htmladdnormallink{http://www.csm.ornl.gov/pvm/}
{http://www.csm.ornl.gov/pvm/}}. It should compile and install without
any problem. You will have to make a few modifications to your {\tt
.cshrc} file, such as defining the {\tt PVM\_ROOT} environment
variable, but this is all explained clearly in the PVM documentation.
Note that all executables (or at least a link to them) must reside in
the {\tt \$PVM\_ROOT/bin/\$PVM\_ARCH} directory in order for parallel
processes to be spawned correctly. The environment variable {\tt
PVM\_ARCH} is set in your {\tt .cshrc} file and contains a string
representing the current architecture type. To run a parallel
application, you must first start up the daemon on each of the
machines you plan to use in the computation. How to do this is also
explained in the PVM documentation.

\paragraph{Communication with Shared Memory.}
\label{shared}
In the shared memory configuration, it is not necessary to use
message passing to move information from one module to another since
memory is globally accessible. In the few cases where the user would
ordinarily have to pass information using message passing, it is
easiest and most efficient to simply copy the information to the new
location. This copying gets done in the {\em send} function and hence
the {\em receive} function is never actually called. This means that
the user must perform all necessary initialization, etc. in the send
function. This makes it a little confusing to write source code which
will work for all configurations. However, the confusion should be
minimized by looking at the sample applications, especially the VRP solver,
which works in all configurations, sequential, distributed parallel, and
shared parallel. 

\paragraph{Configuring the Modules.}
\label{configuration}
In the make file, there are
four variables that control which modules run as separate executables
and which are called directly in serial fashion. The variables are as
follows:
\begin{description}
        \item [COMPILE\_IN\_CG:] If set to {\tt TRUE}, then the cut generator
        function will be called directly from the LP in serial
        fashion, instead of running as a separate executable. This is
        desirable if cut generation is quick and running it in
        parallel is not worth the price of the communication overhead.
        \item [COMPILE\_IN\_CP:] If set to {\tt TRUE}, then the cut
        pool(s) will be maintained as a data structure auxiliary to the
        tree manager. 
        \item [COMPILE\_IN\_LP:] If set to {\tt TRUE}, then the LP
        functions will be called directly from the tree manager. When
        running the distributed version, this
        necessarily implies that there will only be one active
        subproblem at a time, and hence the code will essentially be
        running serially. IN the shared-memory version, however, the
        tree manager will be threaded in order to execute subproblems 
        in parallel.
        \item [COMPILE\_IN\_TM:] If set to {\tt TRUE}, then the tree
        will be managed directly from the master process. This is only
        recommended if a single executable is desired (i.e.~the three
        other variables are also set to true). A single executable is
        extremely useful for debugging purposes.
\end{description}
These variables can be set in virtually any combination, though some
don't really make much sense. Note that in a few user functions that
involve process communication, there will be different versions for
serial and parallel computation. This is accomplished through the use
of {\tt \#ifdef} statements in the source code. This is well documented
in the function descriptions and the in the source files containing
the function stubs. See also Section \ref{shared}.

\paragraph{Executable Names.}
\label{exe_names}
In order to keep track of the various possible configurations, executable and
their corresponding libraries are named as follows. The name of the master
module, along with all other modules compiled in with the master, is set in
the make file. For the other modules, default names are typically used, since
these names have to be hard-coded in order for PVM to correctly spawn the
corresponding processes.  In the fully distributed version, the default names
are \texttt{tm}, \texttt{lp}, \texttt{cg}, and \texttt{cp}. For other
configurations, the executable name is a combination of all the modules that
were compiled together joined by underscores. In other words, if the LP and
the cut generator modules were compiled together (i.e.~{\tt COMPILE\_IN\_CG}
set to {\tt TRUE}), then the executable name would be ``{\tt lp\_cg}'' and the
corresponding library file would be called ``{\tt liblp\_cg.a}.'' You can
rename the executables as you like. However, if you are using PVM to spawn the
modules, as in the fully distributed version, you must set the parameters {\tt
*\_exe} in the parameter file to the new executable names. See Section
\ref{tm_params} for information on setting parameters in the parameter file.

\subsubsection{Microsoft Windows}

First, follow the instructions for compiling SYMPHONY in Section
\ref{getting_started_windows} to ensure you have the proper settings. Once the
stub files in the {\tt SYMPHONY-4.0$\backslash$USER} hierarchy are
filled in, you should be able to compile the new application and run it
successfully. 

\subsection{Debugging Your Application}

Much of this section applies to Unix operating systems. However, it may
also be useful for Windows users.

\subsubsection{The First Rule}

\BB\ has many built-in options to make debugging easier. The most
important one, however, is the following rule. {\bf It is easier to
debug the fully sequential version than the fully distributed
version}. Debugging parallel code is not terrible, but it is more
difficult to understand what is going on when you have to look at the
interaction of several different modules running as separate
processes. This means multiple debugging windows which have to be
closed and restarted each time the application is re-run. For this
reason, it is highly recommended to develop code that can be compiled
serially even if you eventually intend to run in a fully distributed
environment. This does make the coding marginally more complex, but
believe me, it's worth the effort. The vast majority of your code will
be the same for either case. Make sure to set the compile flag to
``{\tt -g}'' in the make file.

\subsubsection{Debugging with PVM}
\label{debugging-PVM}
If you wish to venture into debugging your distributed application,
then you simply need to set the parameter {\tt *\_debug}, where * is
the name of the module you wish to debug, 
to the value ``4'' in the parameter file (the number ``4'' is chosen
by PVM). This will tell PVM to spawn the particular process or
processes in question under a debugger. What PVM actually does in this
case is to launch the script {\tt \$PVM\_ROOT/lib/debugger}. You will
undoubtedly want to modify this script to launch your preferred
debugger in the manner you deem fit. If you have trouble with this,
please send e-mail to the list serve (see Section \ref{resources}).

It's a little tricky to debug interacting parallel processes, but you
will quickly get the idea. The main difficulty is in that the order of
operations is difficult to control. Random interactions can occur when
processes run in parallel due to varying system loads, process
priorities, etc. Therefore, it may not always be possible to duplicate
errors. To force runs that you should be able to reproduce, make sure
the parameter {\tt no\_cut\_timeout} appears in the parameter file or
start \BB\ with the ``{\tt -a}'' option. This will keep the cut
generator from timing out, a major source of randomness. Furthermore,
run with only one active node allowed at a time (set {\tt
max\_active\_nodes} to ``1''). This will keep the tree search from
becoming random. These two steps should allow runs to be reproduced.
You still have to be careful, but this should make things easier.

\subsubsection{Using {\tt Purify} and {\tt Quantify}}

The make file is already set up for compiling applications using {\tt
purify} and {\tt quantify}. Simply set the paths to the executables
and type ``{\tt make pall}'' or ``{\tt p*}'' where * is the module you
want to purify. The executable name is the same as described in
Section \ref{exe_names}, but with a ``p'' in front of it. To tell PVM
to launch the purified version of the executables, you must set the
parameters {\tt *\_exe} in the parameter file to the purified
executable names. See Section \ref{tm_params} for information on
setting parameters in the parameter file.

\subsubsection{Checking the Validity of Cuts and Tracing the Optimal Path}
\label{debugging}
Sometimes the only evidence of a bug is the fact that the optimal
solution to a particular problem is never found. This is usually
caused by either (1) adding an invalid cut, or (2) performing an
invalid branching. There are two options available for discovering
such errors. The first is for checking the validity of added cuts.
This checking must, of course, be done by the user, but \BB\ can
facilitate such checking. To do this, the user must fill in the
function \hyperref{{\tt user\_check\_validity\_of\_cut()}} {{\tt
user\_check\_validity\_of\_cut()} (see Section
}{)}{user_check_validity_of_cut}. THIS function is called every time a
cut is passed from the cut generator to the LP and can function as an
independent verifier. To do this, the user must pass (through her own
data structures) a known feasible solution. Then for each cut passed
into the function, the user can check whether the cut is satisfied
by the feasible solution. If not, then there is a problem! Of course,
the problem could also be with the checking routine. To see how this is
done, check out the sample application file {\tt Vrp/cg\_user.c}.
After filling in this function, the user must recompile everything
(including the libraries) after uncommenting the line in the make file
that contains ``{\tt BB\_DEFINES += -DCHECK\_CUT\_VALIDITY}.'' Type
``{\tt make clean\_all}'' and then ``{\tt make}.''

Tracing the optimal path can alert the user when the subproblem which
admits a particular known feasible solution (at least
according to the branching restrictions that have been imposed so far)
is pruned. This could be due to an invalid branching. Note that this
option currently only works for branching on binary variables. To use
this facility, the user must fill in the function \hyperref{{\tt
user\_send\_feas\_sol()}} {{\tt user\_send\_feas\_sol()} (see Section
}{)}{user_send_feas_sol}. All that is required is to pass out an array
of user indices that are in the feasible solution that you want to
trace. Each time the subproblem which admits this feasible solution is
branched on, the branch that continues to admit the solution is
marked. When one of these marked subproblems is pruned, the user is
notified.

\subsubsection{Using the {\tt Interactive Graph Drawing} Software}
\label{IGD}
The Interactive Graph Drawing (IGD) software package is included with
\BB\ and \BB\ facilitates its use through interfaces with the
package. The package, which is a Tcl/Tk application, is extremely
useful for developing and debugging applications involving graph-based
problems. Given display coordinates for each node in the graph, IGD
can display support graphs corresponding to fractional solutions with or
without edge weights and node labels and weights, as well as other
information. Furthermore, the user can interactively modify the graph
by, for instance, moving the nodes apart to ``disentangle'' the
edges. The user can also interactively enter violated cuts through the
IGD interface.

To use IGD, you must have installed PVM since the drawing window runs
as a separate application and communicates with the user's routines
through message passing. To compile the graph drawing application,
type ``{\tt make dg}'' in the \BB\ root directory. The user
routines in the file {\tt user\_dg.c} can be filled in, but it is not
necessary to fill anything in for basic applications. 

After compiling {\tt dg}, the user must write some subroutines that
communicate with {\tt dg} and cause the graph to be drawn.
Regrettably, this is currently a little more complicated than it needs
to be and is not well documented. However, by looking at the sample
application, it should be possible to see how it is done. To
enable graph drawing, put the line {\tt do\_draw\_graph 1} into the
parameter file or use the {\tt -d} command line option. It can be difficult to
get IGD to work. If you are interested in using it and cannot get it to work,
feel free to contact me.

\subsubsection{Other Debugging Techniques}

Another useful built-in function is MakeMPS, which will write the
current LP relaxation to a file in MPS format. This file can then be
read into the LP solver interactively or examined by hand for errors.
Many times, CPLEX gives much more explicit error messages
interactively than through the callable library. The form of the
function is 
\begin{verbatim}
void MakeMPS(LPData *lp_data, int bc_index, int iter_num)
\end{verbatim}
The matrix is written to the file ``{\tt
matrix.[bc\_index].[iter\_num].mps}'' where {\em bc\_index} is the
usually passed as the index of the current subproblem and {\em
iter\_num} is the current iteration number. These can, however, be any
numbers the user chooses. If \BB\ is forced to abandon solution
of an LP because the LP solver returns an error code, the current LP
relaxation is automatically written to the file ``{\tt
matrix.[bc\_index].[iter\_num].mps}'' where {\em bc\_index} is the
index of the current subproblem and {\em iter\_num} is the current
iteration number. MakeMPS can be called using breakpoint code to
examine the status of the matrix at any point during execution.

Logging is another useful feature. Logging the state of the search tree can
help isolate some problems more easily. See Section \ref{tm_params}
for the appropriate parameter settings to use logging.

\subsection{Controlling Execution and Output}
\label{output}
Calling \BB\ with no arguments simply lists all command-line options.  Most of
the common parameters can be set on the command line. Sometimes, however, it
may be easier to use a parameter file. To invoke \BB\ with a parameter file
type ``{\tt master -f filename ...}'' where filename is the name of the
parameter file. The format of the file is explained in Section
\ref{parameter_file}. 

The output level can be controlled through the use of the verbosity
parameter. Setting this parameter at different levels will cause
different progress messages to be printed out. Level 0 only prints out
the introductory and solution summary messages, along with status
messages every 10 minutes. Level 1 prints out a message every time a
new node is created. Level 3 prints out messages describing each
iteration of the solution process. Levels beyond 3 print out even more
detailed information.

There are also two possible graphical interfaces. For graph-based
problems, the Interactive Graph Drawing Software allows visual display
of fractional solutions, as well as feasible and optimal solutions
discovered during the solution process. For all types of problems,
VBCTOOL creates a visual picture of the branch and cut tree, either
in real time as the solution process evolves or as an emulation from a
file created by
\BB. See Section \ref{tm_params} for information on how to use VBCTOOL
with SYMPHONY. Binaries for VBCTOOL can be obtained at \\ 
{\tt \htmladdnormallink
{http://www.informatik.uni-koeln.de/ls\_juenger/projects/vbctool.html}
{http://www.informatik.uni-koeln.de/ls\_juenger/projects/vbctool.html}}.


\subsection{Other Resources}
\label{resources}
There is a \BB\ user's list serve for posting questions/comments.
To subscribe, send ``{\tt subscribe symphony-users}'' to
{\tt \htmladdnormallink{majordomo@branchandcut.org}
{mailto:majordomo@branchandcut.org}}. There is also a Web site for
\htmladdnormallink{SYMPHONY}{http://branchandcut.org/SYMPHONY} 
\begin{latexonly}
at {\tt http://branchandcut.org/SYMPHONY}
\end{latexonly}.  
Bug reports can be sent to \\
{\tt \htmladdnormallink{symphony-bugs@branchandcut.org}
{mailto:symphony-bugs@branchandcut.org}}.


